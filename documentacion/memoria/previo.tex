% -*-previo.tex-*-
% Este fichero es parte de la plantilla LaTeX para
% la realización de Proyectos Final de Carrera, protejido
% bajo los términos de la licencia GFDL.
% Para más información, la licencia completa viene incluida en el
% fichero fdl-1.3.tex

% Copyright (C) 2009 Pablo Recio Quijano 

\section*{Agradecimientos}

Me gustaria agradecer y dedicar este texto a:
\begin{itemize}
\item Manuel Palomo por su labor de tutorización y consejos.
\item Javier Cadenas por su asesoramiento respecto al diseño de videojuegos orientado al género de las aventuras gráficas.
\item José Joaquín Rodríguez por sus consejos y sugerencias sobre cuales libros escoger para la documentación histórica.
\item Celia Fermoselle, \cursiva{boredBit}, Laura J. Torres, Daniel Brey y Encarnación M. R. porque sin su colaboración este proyecto no tendría la calidad que tiene. 
\item Eric Juste por ser mi querido \cursiva{betatester} y apoyo moral durante la realización del proyecto.
\item Toda mi familia por su cariño incondicional durante todo este tiempo.
\item La comunidad de Unity3D por ayudarme a resolver dudas y consejos sobre la implementación.
\item La comunidad de ZehnGames, Games Tribune y muchos más por su labor de difusión de mi proyecto, sin ella no habría conseguido a mis colaboradores.
\end{itemize}

\cleardoublepage

\section*{Licencia} % Por ejemplo GFDL, aunque puede ser cualquiera

Este documento ha sido liberado bajo Licencia GFDL 1.3 (GNU Free
Documentation License). Se incluyen los términos de la licencia en
inglés al final del mismo. Sin embargo, las imágenes de videojuegos comerciales incluidos están sujetos a copyright y no se distribuyen bajo licencia libre.\\

Copyright (c) 2014 Alicia Guardeño Albertos.\\

Permission is granted to copy, distribute and/or modify this document under the
terms of the GNU Free Documentation License, Version 1.3 or any later version
published by the Free Software Foundation; with no Invariant Sections, no
Front-Cover Texts, and no Back-Cover Texts. A copy of the license is included in
the section entitled "GNU Free Documentation License".\\

\cleardoublepage

\section*{Notación y formato}
Para poder mantener un estilo uniforme y legible, a lo largo de esta memoria de \textbf{Proyecto Final de Carrera} se ha utilizado la siguiente notación:
\begin{itemize}
\item Para referirnos a nombres de ficheros o funciones de un lenguaje usaremos: \comando{unity.cs}.
\item Cuando mencionemos el nombre de un programa se hará de la siguiente manera: \programa{Unity}.
\item Los nombres de los objetos de \programa{Unity3D} se escriben con este formato: \gameobject{GameController}.
\item Los componentes de los objetos, excepto cuando se traten de \emph{Scripts}, tendrán el siguiente formato: \gocomponent{Rigidbody}.
\item En el caso de adjuntar un fragmento de código, utilizamos bloques como el siguiente:
\end{itemize}

\lstinputlisting[style=C++]{./codigo/hola_mundo.cs}