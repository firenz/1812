\newpage

En este apartado, hay que hacer una pequeña definición del proyecto,
de que se pretende con él, y que motivaciones nos han llevado a
realizar este proyecto.\\

Vamos a aprovechar para introducir los elementos tipo
\comando{float}. Estos son elementos como imágenes o tablas, que
queremos independizar un poco del resto del documento, además de
añadirle una referencia propia, una pequeña descripción, o una
numeración para que se muestre en los índices.\\

A continuación se muestran dos imágenes iguales, una colocada
directamente con el comando que vimos antes
\comando{$\backslash$includegraphics}, y la siguiente con el entorno
apropiado:

\begin{center}
\includegraphics[scale=0.5]{mybob-calimetux.png}
\end{center}

Ahora como se debe hacer:

\begin{figure}[H] % La opción H indica que queremos que la imagen esté
                  % ahí forzosamente. Hay muchas más opciones, como
                  % 'h', que indica que si es posible la coloque tras
                  % la última linea de texto, si no, la colocará en el
                  % mejor hueco que encuentre durante la
                  % compilación. Otras son 't' (al comienzo de la
                  % página), 'b' (al final de la página) ...
  
  \label{cal-tux} % Al comienzo, así al marcar en una referencia vemos
                  % la imagen.
  \begin{center}
    \includegraphics[scale=0.5]{mybob-calimetux.png}
  \end{center}
  \caption{Calimero versión libre}
\end{figure}

Además de que el resultado es igual, ahora podemos referenciar la
imagen, usando la referencia definida en el entorno \comando{figure},
de forma \ref{cal-tux}. \\

Si observamos que en \negrita{Índice de figuras}, se ha incluido
la imagen. Por tanto, a partir de ahora usaremos este entorno.\\

De forma similar, podemos hacer tablas. Recordemos la tabla del
ejemplo básico:

\begin{table}[H]
  \label{metal}
  \begin{center}
  \begin{tabular}{| c ||m{2.2cm}|m{2.2cm}|m{2.2cm}|m{2.2cm}|}
    \hline
    Nombre del grupo & Vocalista & Guitarra & Bajo & Bateria\\
    \hline
    Metallica & James Hetfield & Kirk Hammet & Robert Trujillo & Lars
    Ulrich\\
    \hline
    Guns N' Roses & Axl Rose & Robin Finck & Tommy Stinson & Brian
    Mantia \\
    \hline
    Queen & Freddie Mercury (RIP) & Brian May & John Deacon & Roger
    Taylor\\
    \hline
    AC/DC & Brian Johnson & Angus y Malcom Young & Cliff Williams &
    Phil Rudd\\
    \hline
    Black Label Society & Zakk Wylde & Zakk Wylde y Nick Catanese &
    John DeServio & Craig Nunenmacher\\
    \hline
  \end{tabular}
\end{center}
\caption{Grupos significativos en el Rock, Heavy y Metal}
\end{table}

El resultado de la tabla \ref{metal} es idéntico. Ademas como podemos
ver, al usar otro entorno, se nos genera en el otro índice,
numerándose de forma independiente. Una de las grandes versatilidades
de \LaTeX