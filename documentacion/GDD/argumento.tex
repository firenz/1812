En este apartado detallaremos el argumento y el guión del juego. Al ser \nombrejuego un videojuego del género de las aventuras gráficas, estos tienen más importancia, pues a través de la narración vamos conociendo a los personajes y a cómo resolver los puzzles que nos deparará en el juego. 
    \section{Argumento y narrativa}
        \subsection{Trasfondo argumental}
        Hace un par de días, el Protagonista ha suspendido un examen con la infame nota de un 4,9. Ante tal nota, el Protagonista se dispone ir a la revisión del examen al despacho del profesor. Lamentablemente el profesor no se encuentra allí, pero cómo nuestro Protagonista es un cabezota, hará todo lo posible para encontrar al profesor y convencerle de que merece el aprobado.
        
        \subsection{Elementos clave}
        \subsection{Secuencias cinemáticas}
        En esta sección detallaremos las secuencias cinemáticas que habrá en \nombrejuego. Estos son momentos del juego en el que el \emph{Jugador} no podrá controlar al Protagonista y el juego tomará control de la situación. En juegos grandes se suele invertir una cantidad enorme en hacer las secuencias cinemáticas lo más vistosas posibles empleando en su mayoría vídeos. Juegos más pequeños prefieren evitar estas escenas por el enorme gasto que suponen, y se las ingenian en hacerlos de otras maneras.
        
        En \nombrejuego al ser también un juego de pequeña envergadura, la cantidad de secuencias cinemáticas del juego es mínima y sólo habrá una al principio de entrar en un nuevo escenario, y otra después de resolver un puzzle. Por ahora solo hay 
            
    \section{Mundo del juego}
        \subsection{Estilo visual general}%General Look and Feel
        \subsection{Primer escenario}
        \subsection{Segundo escenario}
        %...
        
    \section{Personajes}
    En esta sección definiremos brevemente a los personajes que intervendrán en el juego. Servirá sobretodo a la hora de definir las personalidades y reflejarlas en la manera de hablar de estos en el guión.
    
        \subsection{Protagonista}
            \subsubsection{Trasfondo}
            Estudiante del Grado de Historia en la Universidad de Cádiz, ha suspendido un examen con un 4,9 y hará todo lo posible para convencer al profesor de que merece el aprobado.
            \subsubsection{Personalidad}
            De personalidad humorística y despreocupada, con cierta aversión al profesor que le ha suspendido.
            \subsubsection{Aspecto}
            Pelo negro y desaliñado, con barbita dejada tal y cómo se lleva ahora entre los universitarios. Con sudadera y vaqueros.
                
        \subsection{Profesor}
            \subsubsection{Trasfondo}
            Profesor de una asignatura en el Grado de Historia en la Universidad de Cádiz, no se sabe donde está.
            \subsubsection{Personalidad}
            Despistado y afable, pero es muy estricto con los alumnos.
            \subsubsection{Aspecto}
            Típico profesor mayor con barba y traje.
            
        \subsection{Bibliotecaria}
            \subsubsection{Trasfondo}
            Es la bibliotecaria de la Facultad de Filosofía y Letras desde hace muchos años, ya los años les pasa factura y se olvida de las cosas o no puede llevar los libros con tanta facilidad.
            \subsubsection{Personalidad}
            Olvidadiza y muy estricta con las normas de la biblioteca, sobretodo con la de mantener el silencio.
            \subsubsection{Aspecto}
            Mujer mayor con gafas de lectura antigua.
    
    Estos son los personajes definidos por ahora, más adelante se irán añadiendo más y así sucesivamente hasta completar con todo el elenco de personajes que harán su aparición en el juego.
        %...