 \section{Organización de los recursos dentro del proyecto}
    Que haya una buena organización de los recursos facilita mucho el trabajo a los desarrolladores a la hora de tener que implementarlo dentro del juego, y así no tener que buscarlo en un sinfín de carpetas desorganizadas. Por ello es importante establecer una jerarquía en el orden de los recursos y de las carpetas que lo contengan. La organización de los recursos dentro del repositorio de \nombrejuego es la siguiente:
    
    \begin{itemize}
    \item \negrita{Recursos}/
        \begin{itemize}
        \item Audio/
            \begin{itemize}
            \item Música/
            \item FX/
            \end{itemize}
        \item Gráficos/
            \begin{itemize}
            \item GUI/
                \begin{itemize}
                \item Pantalla 1/
                \item Pantalla 2/
                \item ...
                \end{itemize}
            \item HUD/
            \item Personajes/
                \begin{itemize}
                \item Personaje 1/
                \item Personaje 2/
                \item ...
                \end{itemize}
            \item Escenarios/
                \begin{itemize}
                \item Escenario 1/
                \item Escenario 2/
                \item ...
                \end{itemize}
            \item Objetos/
                \begin{itemize}
                \item Objeto 1/
                \item Objeto 2/
                \item ...
                \end{itemize}
            \end{itemize}
        \end{itemize}
    \end{itemize}
    
    \section{Lista de recursos}
    Aquí haremos una lista que se irá actualizando con los recursos audiovisuales necesarios para poder realizar \nombrejuego. Así no se hará ningún trabajo de más o de menos, y que en el caso de tener que realizar modificaciones, estas sean pocas.
    
        \subsection{Arte}
        Todos los recursos gráficos que necesitará el juego, por ahora son los comentados en las siguientes secciones y sub-secciones.
            \subsubsection{Lista de sprites y fondos}
            Los \emph{sprites} son el nombre con el que se llama a los dibujos pixelados con los que se hacen los gráficos 2D para los videojuegos, si bien los fondos son sprites es mejor llamarlos de manera diferente, el motivo principal es porque existe la diferencia de que un fondo y un sprite es su tamaño. Pues un fondo ocupa toda la pantalla, mientras que un sprite es mediano o pequeño, pero nunca ocupa la pantalla en su totalidad.
            Lista de sprites de personajes:
            \begin{itemize}
            \item Protagonista.
            \item Bibliotecaria.
            \item Profesor.
            \item Escritor fracasado en su mesa de firma de libros.
            \item Otro profesor.
            \item Vendedor de la tienda de recuerdos.
            \end{itemize}
            
            Lista de sprites de objetos:
            \begin{itemize}
            \item Examen con la nota de 4,9 puntos en rojo.
            \item Banderitas de España y Francia.
            \item Nota de biblioteca.
            \end{itemize}
            
            \subsubsection{Lista de animaciones}
            \begin{itemize}
            \item Protagonista caminando.
            \item Protagonista hablando.
            \item Protagonista cogiendo objeto del suelo y metiéndoselo en el bolsillo de la sudadera.
            \item Protagonista cogiendo un objeto a su nivel y metiéndoselo en el bolsillo de la sudadera.
            \item Protagonista mirando el móvil.
            \item Nota de papel cayendo en el suelo.
            \end{itemize}
            
            \subsubsection{Lista de efectos}
            En este juego no hay efectos.
            
            \subsubsection{Lista de arte de la interfaz}
            \begin{itemize}
            \item Icono de lupa para el ratón.
            \item Icono de lupa con borde blanco (u otro elemento que destaque) para indicar que hay un objeto con el que se puede interactuar.
            \item Icono de bocadillo de cómic para el ratón.
            \item Icono de flecha para cambiar o salir de escenario. 
            \item Fondo con un \emph{smartphone} negro y táctil, que se vea la mano del protagonista cómo si fuera a pulsar algo en el \emph{smartphone}.
            \item Icono de \emph{smartphone} pequeño para la pantalla de juego.
            \item Icono de aplicación de \emph{smartphone} de mapas como si fuera el Google Maps.
            \item Icono de aplicación de \emph{smartphone} de notas.
            \item Icono de aplicación de \emph{smartphone} para guardar partidas.
            \item Icono de aplicación de \emph{smartphone} de Gadipedia como si fuera la Wikipedia.
            \item Icono de aplicación de \emph{smartphone} de ajustes.
            \item Icono de aplicación de \emph{smartphone} para salir del juego y volver al menú principal.
            \item Fondo de \emph{smartphone} plano de varios colores.
            \item Botón de aplicación de \emph{smartphone} reutilizable (que se pueda alargar y ensanchar).
            \item Cuadro de texto de aplicación de \emph{smartphone} reutilizable (que se pueda alargar y ensanchar).
            \item Barra de scroll de aplicación de \emph{smartphone} reutilizable (que se pueda alargar).
            \item Mapa parcial y sencillo en dos colores planos de Cádiz para la aplicación de mapas (similar al Google Maps).
            \item Botón circular o banderita pequeña para indicar los lugares a los que podemos ir en el mapa.
            \end{itemize}
            
            \subsubsection{Lista de escenas cinemáticas}
            Solo habrá una escena cinemática, la de los créditos, que será realizada con la composición de otras imágenes estáticas y se irán cambiando. Dichas escenas cinemáticas, por ahora con posibilidad de modificación, son:
            \begin{itemize}
            \item Protagonista dándose la mano con el profesor.
            \item Protagonista estudiando en la biblioteca mientras pasa la bibliotecaria.
            \item ...
            \item Examen con un 4,9 tachado y un 5 en rojo, sobre una mesa.
            \end{itemize}
            
        \subsection{Sonido}
        Todos los recursos auditivos que necesitará el juego, por ahora son los comentados en las siguientes secciones y subsecciones.
            \subsubsection{Sonido ambiental}
            \begin{itemize}
            \item Racha de viento que sopla por la ventana del despacho.
            \item Murmullo de gente hablando y luego un carraspeo de garganta de la bibliotecaria (una señora mayor) para mandar a callar a la gente.
            \end{itemize}
            \subsubsection{Sonidos al realizar interacciones}
            \begin{itemize}
            \item Cerrar ventana.
            \item Coger papel del suelo.
            \item Coger chinchetas (u objetos semimetálico) del suelo.
            \end{itemize}
            \subsubsection{Sonidos de interfaz}
            \begin{itemize}
            \item Pulsación de móvil.
            \end{itemize}
            
        \subsection{Música}
        Todos los recursos musicales que necesitará el juego, por ahora son los comentados en las siguientes secciones y subsecciones.
            \subsubsection{Menú principal}
            Para esta pantalla una canción más elaborada vendría bien, no importa si es de tono desenfadado o más clásica.
            \subsubsection{Ambiente}
            Sería recomendable hacer una canción para cada escenario, todas ellas de tono desenfadado y con toque cómico. Que fueran cortas y se  puedan poner en bucle. 
            
            Aparte de esas canciones, vendría bien una que fuera más como un tono de victoria, que se pondría cada vez que se resuelva un puzzle y podemos ir al siguiente escenario.
            \subsubsection{Créditos}
            Lo mismo que en la de Menú principal, una canción más elaborada. La intención con esta canción es darle al \emph{Jugador} la sensación de satisfacción de haberse pasado el juego.