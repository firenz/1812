\documentclass[a4paper,11pt,titlepage]{book}
%\usepackage{filecontents}
%\usepackage[style=authoryear,backend=bibtex]{biblatex}
% Necesario para poder hacer funcionar la bibliografia hecha en BibTeX

\usepackage{./estilos/estiloBase} % Basicamente son todas las
                                  % librerias usadas. En caso de que
                                  % falten librerias se van añadiendo
                                  % al fichero.
\usepackage{./estilos/colores}  % Algunos colores ya generados, para
                                % los algunos estilos más avanzados.
\usepackage{./estilos/comandos} % Algunos comandos personalizados

\graphicspath{{./imagenes/}} % Indicamos la ruta donde se encuentran
                             % las imagenes, para ahorrarnos la ruta
                             % completa, y solo modificar aquí si en
                             % un momento dado lo movemos
%\ifnum \@itemdepth >4\@toodeep\else
%---------------------------------------------------------------------------------

\title{{\Huge \bf 1812: La aventura}\\{\large \em Documento de diseño}\\{\normalsize Versión 0.1}}
\author{Alicia Guardeño Albertos}
\date{\today}

\begin{document}


% Renombramos las figuras y las tablas
\renewcommand{\figurename}{Figura}
\renewcommand{\listfigurename}{Indice de figuras}
\renewcommand{\tablename}{Tabla}
\renewcommand{\listtablename}{Indice de tablas}

\vfill
\maketitle
\vfill
%\newpage
%\frontmatter
    \tableofcontents
    \listoffigures
    %\listoftables

\setlength{\parskip}{\baselineskip} %Lo pongo aqui para evitar el salto de línea en los índices
%\mainmatter

%\newpage
\chapter{Historial de cambios}
\label{chap:historial}
En este capítulo se informará de todas las modificaciones que ha sufrido este documento, indicando la fecha y una descripción de los cambios más destacados.

Por ahora no hay revisiones al ser la primera versión.

%\newpage
\chapter{Introducción}%Game overview
\label{chap:introduccion}
% -*-cap1.tex-*-
% Este fichero es parte de la plantilla LaTeX para
% la realización de Proyectos Final de Carrera, protejido
% bajo los términos de la licencia GFDL.
% Para más información, la licencia completa viene incluida en el
% fichero fdl-1.3.tex

% Copyright (C) 2009 Pablo Recio Quijano 



\section{Introducción}
\nombrejuego es un videojuego en 2D del género de las aventuras gráficas, con una ambientación en la Cádiz actual cuyo objetivo es tanto de entretener al jugador como que aprenda anécdotas y hechos relacionados con Cádiz en los años que fue asediada y vieron nacer la Constitución de 1812. Dicho videojuego hace uso de \programa{Unity3D}, un motor de desarrollo de videojuegos multiplataforma. 

En \nombrejuego controlaremos a un estudiante de la Licenciatura de Historia de la Universidad de Cádiz, que recientemente ha suspendido un examen y va al despacho a pedir una revisión para su nota. Sin embargo, no logra encontrar al profesor y se embarcará en una aventura para descubrir el paradero de dicho profesor ayudándose de las pistas obtenidas al resolver puzzles teniendo estos siempre una relación con la Cádiz de 1812. El videojuego tiene que estar completamente documentado, pues una gran gran parte de la jugabilidad y de la parte educativa del juego, recae completamente en el buen diseño que se haga de los puzzles, estancias e interacciones con otros personajes no controlables.

Con todo esto se pretende demostrar que un videojuego se puede usar de material didáctico en entidades educativas de todos los ámbitos, y promover el contenido interactivo sin olvidarse del entretenimiento como manera de aprendizaje para jugadores de cualquier edad.

\section{Contexto}
\label{sec:contexto}

[...]

\subsection{Historia del auge, caída, y resurgir de las aventuras gráficas}
\label{sec:ag}
La aventura gráfica es un subgénero de los videojuegos de aventura. Su mecánica consiste en ir avanzando por el mundo, escenario o juego a través de la resolución de diversos puzzles, planteados como situaciones que se suceden en la historia, interactuando con personajes y objetos a través de un menú de acciones o interfaz similar, utilizando un cursor para manejar al personaje y realizar las distintas acciones. En su concepción clásica, esta siempre incluía la visión de los personajes en tercera persona, aunque en varias ocasiones se planteasen en primera persona. 

Este género se originó a partir de las aventuras conversacionales anteriores a los años 80. Era una época en la que los ordenadores personales aún carecían de gráficos y sólo se podía interactuar con ellos escribiendo líneas de comandos a lo que te contestaban en texto. 

Como se estaban instaurando con éxito en los hogares, pronto hubo gente que se dedicó a buscar un nuevo abanico de entretenimiento y ocio basado en ellos. De ellos nacieron las aventuras conversacionales, la primera de ellas fue Colossal Cave Adventure \cita{cave-adventure} (véase la figura ~\ref{fig:text-adventure}). En estos la acción se desarrollaba describiendo en un párrafo la situación actual del protagonista y abajo un cuadro de texto en el que había que escribir sencillas frases para interactuar con el entorno, del tipo ``hablar con el anciano'', ``usar llave'', ``salir por la puerta'' o indicando puntos cardinales para ir de un sitio a otro, como por ejemplo ``norte'' o ``sur''.

\begin{figure}[H] 
  \begin{center}
    \includegraphics[scale=0.5]{first-text-adventure.png}
  \end{center}
  \caption{Colossal Cave Adventure (PDP-10, 1977), la primera aventura conversacional}
    \label{fig:text-adventure}
\end{figure}

No fue hasta en 1980, en la que la compañía Online System (que más tarde pasó a llamarse Sierra Online \cita{sierra-online}) creó la primera aventura gráfica propiamente dicha. Este juego con gráficos rudimentarios era Mystery House \cita{mystery-house} (Apple II, 1980) (veáse la figura ~\ref{fig:mystery-house}), luego le siguió Wizard and the Princess \cita{wizard-princess} (Apple II, 1980) ya con gráficos en color, y finalmente establecieron el género con King's Quest \cita{king-quest} (Apple II, 1984) con su aparición en varios sistemas y asentándose en la industria con cada vez títulos más fuertes.

\begin{figure}[H] 
  \begin{center}
    \includegraphics[scale=1]{mystery-house-first-graphic-adventure.png}
  \end{center}
  \caption{Mystery House, la primera aventura con gráficos}
  \label{fig:mystery-house}
\end{figure}

El lanzamiento del Apple Macintosh y su interfaz controlada por ratón supuso la creación de las aventuras gráficas \cursiva{point-and-click}, introduciendo a más empresas a este mercado. LucasArts \cita{lucasarts} fue una de ellas, la cual consiguió un éxito rotundo con su primer juego, Maniac Mansion (veáse la figura ~\ref{fig:maniac-mansion}) \cita{maniac-mansion}, con una interfaz íntegramente \cursiva{point-and-click}, imponiéndose como otra gran empresa dentro de la industria. Mientras tanto, Sierra Online también se sumaría a este cambio pasando su sistema de introducir comandos por una rudimentaria interfaz \cursiva{point-and-click} con Manhunter: New York \cita{manhunter} (veáse la figura ~\ref{fig:manhunter}).

\begin{figure}[H] 
  \begin{center}
    \includegraphics[scale=1]{maniac-mansion-commodore64.png}
  \end{center}
  \caption{Maniac Mansion (1987), con su interfaz exclusivamente \cursiva{point-and-click}}
    \label{fig:maniac-mansion}
\end{figure}

\begin{figure}[H] 
  \begin{center}
    \includegraphics[scale=0.7]{manhunter.png}
  \end{center}
  \caption{Manhunter: New York (1989), con una interfaz \cursiva{point-and-click} primitiva con teclas}
    \label{fig:manhunter}
\end{figure}

Sierra Online y LucasArts seguían un camino lleno de rivalidades la una con la otra, aunque cada una seguiría su propio camino:
\begin{itemize} 
\item Sierra Online apostaba por las grandes sagas como King's Quest \cita{king-quest-saga} o Space Quest \cita{space-quest}, por otro  lado, LucasArts prefería las aventuras sin continuidad por estos años.

\item Los juegos de Sierra Online mantenían un desarrollo no encadenado, no obligaban a realizar una acción en concreto para poder avanzar en el juego. En cambio, los de LucasArts tenían un planteamiento más lineal en el que sólo se podía avanzar hasta cierto límite sin realizar la acción precisa.

\item En relación al punto anterior, las aventuras de LucasArts eran mucho más sencillas de finalizar, y por extensión, gozaban de mayor popularidad.

\item Sierra Online apostaba por una perspectiva quizás más adulta, con historias que rozaban la épica, como en King's Quest, mientras que LucasArts era similar a su matriz cinematográfica, con un tono más apto para todos los públicos al ser de tono aventurero y humorístico.
\end{itemize}

Si bien estas fueron las dos desarrolladoras más destacadas en el género, hubo varios juegos de otras compañías que también son dignos de destacar. Entre otros, Policenauts \cita{policenauts} (1994) de Hideo Kojima, u otras centradas en el terror, tales como Shadow of the Comet \cita{shadow-comet} (véase la figura ~\ref{fig:shadow-comet}), o como la saga Clock Tower \cita{clock-tower} de Human Entertainment \cita{human-entertainment}. Myst \cita{myst} (1993) de corte fantástico, que poseía una perspectiva en primera persona en contrapunto a la tercera persona que se estilaba, tuvo una gran recepción en el público. 

\begin{figure}[H] 
  \begin{center}
    \includegraphics[scale=1]{shadow-comet.png}
  \end{center}
  \caption{Shadow of the Comet (DOS, 1993), realizado por la compañia francesa Infogrames}
    \label{fig:shadow-comet}
\end{figure}

Sin embargo, no sería hasta 1989, con el lanzamiento de Indiana Jones and the Last Crusade: The Graphic Adventure \cita{indiana-jones} de LucasArts, cuando se pondría este género de moda. Llegó la \cursiva{Edad de Oro} de las aventuras gráficas. Con la aparición de los CD-ROM se podían crear aventuras cada vez más largas y con mejores gráficos, incluso algunos incorporaban elementos 3D pre-renderizados y vídeos de imagen real.Grandes aventuras, en su gran mayoría de LucasArts, marcaron este inicio de la década de los 90, tanto que algunas acabaron como iconografía de la cultura popular y en los anales de la historia de los videojuegos. Una de las más destacadas fue The Secret of Monkey Island \cita{monkey-island} (1990) al centrarse más en la exploración y que el protagonista no pueda morir. Otra fue Day of the Tentacle \cita{day-tentacle} (1993), secuela de Maniac Mansion, que logró tal éxito que con las ganancias le permitió al director del apartado de diseño, Tim Schafer \cita{tim-schafer}, producir el videojuego exitoso Full Throttle \cita{full-throttle} incorporando las voces de Roy Conrad y Mark Hamill.

Sierra Online, por su lado, seguía con sus sagas populares Space Quest y King's Quest. Les mejoró su interfaz \cursiva{point-and-click} por una más amigable para el jugador, sin necesidad de introducir texto. No obstante, algunas de sus aventuras más reconocidas no llegaron a ser las de estas sagas, tal como pasó con Leisure Suit Larry in the Land of the Lounge Lizards \cita{larry-lizards} (1987, remake: 1991) que ganó en 1987 el premio al ``Mejor Juego de Aventura'' de la \cursiva{Software Publishers Association}.

%\begin{figure}[H]
%  \begin{center}
%    \includegraphics[scale=1]{larry-lizards.png}
%  \end{center}
%  \caption{Leisure Suit Larry in the Land of the Lounge Lizards (1987, remake: 1991)}
%    \label{fig:larry-lizards}
%\end{figure}

Lamentablemente, esta \cursiva{Edad de Oro} duró pocos años. A finales de los años 90, el público empezó a desviar sus miradas con las mejoras gráficas y de jugabilidad en los ordenadores. Aparecieron juegos dedicados más a la acción como los shooters en primera persona, y con el asentamiento de internet en los primeros hogares, los juegos online. Las aventuras gráficas poco podían hacer con estos avances, pues su mecánica hace que sean irrelevantes, y su popularidad y ventas cayeron. Así que los editores fueron cada vez más reacios a financiar aventuras gráficas por miedo a malas ventas.

Tal fue la crisís, que Sierra Online casi cerró completamente y LucasArts dejó de publicar después del año 2000. Hubo unas cuantas perlas antes del declive, Broken Sword: The Shadow of the Templars \cita{broken-sword} (1996) o Grim Fandango \cita{grim-fandango} (véase la figura ~\ref{fig:grim-fandango}), siendo este último un fracaso en ventas a pesar de ser aclamado por la crítica y algunos videojuegos más que no mencionaremos.

%\begin{figure}[H] 
%  \begin{center}
%    \includegraphics[scale=1.1]{broken-sword.jpg}
%  \end{center}
%  \caption{Broken Sword: The Shadow of the Templars (1996), una gran aventura gráfica en las que sus secuelas no estuvieron a su altura}
%    \label{fig:broken-sword}
%\end{figure}

\begin{figure}[H] 
  \begin{center}
    \includegraphics[scale=1]{grim-fandango.jpg}
  \end{center}
  \caption{Grim Fandango (1998), hecha íntegramente en 3D sin la clásica interfaz \cursiva{point-and-click}}
    \label{fig:grim-fandango}
\end{figure}

Después, si bien de manera independiente o amateur se fueron haciendo pequeñas obras gracias al auge de Adobe Flash y su soporte en internet. El subgénero de "escapa de la habitación" fue el que más juegos en su haber tuvo en estos años. Sin embargo, eran muy cortas, rudimentarias y repetitivas, no llegando casi nunca al público. 

No fue hasta la llegada de otras nuevas tecnologías, el resurgimiento de este género. La Nintendo DS, y Wii, permitía interactuar con el juego de una manera similar a usar un ratón de ordenador. Como resultado, varios desarrolladores crearon nuevas aventuras gráficas para estas plataformas. Ejemplos de aventuras gráficas de estas plataformas incluyen Zack \& Wiki: Quest for Barbaro's Treasure \cita{zack-wiki} (2007) para Wii, Hotel Dusk: Room 215 \cita{hotel-dusk} (2006) para Nintendo DS, y un port de Broken Sword: The Shadow of the Templars (2009) también para Nintendo DS. 

\begin{figure}[H] 
  \begin{center}
    \includegraphics[scale=0.7]{hotel-dusk.png}
  \end{center}
  \caption{Hotel Dusk: Room 215 (2006), una aventura estilada como si fuera una novela negra}
    \label{fig:hotel-dusk}
\end{figure}

Pero sin duda, el verdadero renacer fue gracias a internet por asentarse totalmente en los hogares, y las mejoras de su velocidad a lo largo de estos años. Al fin era factible el poder promocionarte y distribuir tu juego sin costes intermedios, al menos dentro de un mercado de nicho. Una nueva compañía llamada Telltale Games \cita{telltale}, formada por antiguos miembros de LucasArts, empezó a producir nuevas aventuras gráficas para ordenador. Siguiendo una metodología de distribuir sus juegos de manera episódica, sus juegos incluyen Sam \& Max Save the World \cita{sam-max} (2006), Strong Bad's Cool Game for Attractive People \cita{strong-bad} (2008), el resurgir de Monkey Island con Tales of Monkey Island \cita{tales-monkey} (2009), Back to the Future: The Game \cita{back-future} (2010).

Su ópera prima llegó con el videojuego The Walking Dead \cita{walking-dead} (véase la figura ~\ref{fig:walking-dead}), aclamado tanto por la crítica como con el público. Su sistema de realizar decisiones difíciles en el momento y ver como influían en los personajes impactó enormemente, tanto que mucha gente hizo vídeos en Youtube de sus reacciones y elecciones, fomentando enormemente su difusión y su venta tanto en ordenador como en consolas.

\begin{figure}[H] 
  \begin{center}
    \includegraphics[scale=0.7]{walking-dead.jpg}
  \end{center}
  \caption{The Walking Dead (2012), su estética de cómic americano y su sistema de decisiones cautivó al público}
    \label{fig:walking-dead}
\end{figure}

No hay que olvidarse de otros muchos estudios independientes. Algunos optaron por seguir con las mecánicas clásicas, Machinarium \cita{machinarium} (2009) y Botanicula \cita{botanicula} (2012) de Amanita Designs son ejemplos de ello. Otras por mezclar mecánicas y nuevas tecnologías para reinventar las aventuras gráficas, como Dreamfall \cita{dreamfall} (2006), Portal \cita{portal} (2007) y muchos otros juegos, borrando las líneas del género. También podríamos incluir en este último, las películas interactivas por su semejanza con las aventuras gráficas, la más destacada Heavy Rain \cita{heavy-raiin} (2010) que fue un éxito de ventas.

Actualmente, sobre todo estos dos últimos años, el género está gozando una \cursiva{Edad de Plata}. Las redes nuevamente crearon un nuevo sistema de financiación, el crowdfunding (véase la siguiente sección), y se afianzaron las plataformas de distribución de videojuegos digitales, reforzando la presencia de estudios independientes y la creación de juegos que antes no se hubieran podido permitir. Una de las primeras en aprovecharse de este método, fue Double Fine Productions \cita{double-fine}. Junto con Tim Schafer, en febrero de 2012 lanzó una campaña en Kickstarter, la web de crowdfunding más famosa del mundo, para financiar Broken Age \cita{broken-age} (véase la figura ~\ref{fig:broken-age}). Causó un gran impacto, recaudando hasta la inmensa cantidad de 3,45 millones de dólares, confirmando el establecimiento del crowdfunding como una alternativa viable para financiar proyectos.

Una vez allanado el camino, otros estudios independientes siguieron la estela de Double Fine. AI Lowe, original creador de Leisure Suit Larry, lanzó una campaña para financiar un remake completo de su primera obra \cita{larry-reloaded}. 

\begin{figure}[H] 
  \begin{center}
    \includegraphics[scale=0.2]{broken-age.jpg}
  \end{center}
  \caption{Broken Age (2014), un cuento convertido en aventura gráfica financiada gracias al micromecenazgo}
    \label{fig:broken-age}
\end{figure}

Día a día, más estudios y empresas independientes se animan a lanzar sus aventuras gráficas. Y así gracias a las nuevas tecnologías y los estudios independientes, hemos recuperado este género olvidado.

¿Y qué pasó durante todos estos largos años en España? Pues en 1994 debutó Pendulo Studios \cita{pendulo} con la primera aventura gráfica española, Igor Objetivo Uikokahonia \cita{igor}. En 1997, lograría su mayor renombre con Hollywood Monsters \cita{hollywood-monsters}. Actualmente, son el mayor exponente del género en España con la saga Runaway \cita{runaway}. En 2011 ve la luz Hollywood Monsters 2 \cita{hollywood-monsters-2}, realizada enteramente en alta definición. Mientras que todos estos títulos siguen el más puro estilo clásico del género, el 29 de marzo de 2012, Péndulo lanzó New York Crimes \cita{new-york}, una aventura mucho más oscura y adulta de lo que sigue siendo el estilo clásico de las aventuras gráficas,  y la fusiona con la estética y composición propias del cómic.

\begin{figure}[H] 
  \begin{center}
    \includegraphics[scale=0.5]{hollywood-monsters.jpg}
  \end{center}
  \caption{Hollywood Monsters, la obra más remarcable de Pendulo Studios}
    \label{fig:hollywood-monsters}
\end{figure}

Aún así, pequeños estudios independientes españoles están empezando a hacer sus  pesquisas en este género. Dead Synchronicity \cita{dead-synchronicity} de Fictiorama Studios logró financiarse en abril del 2014, es un gran ejemplo de ello.

\subsection{Videojuegos en el ámbito educacional}
Un juego educativo (o ``juegos serios'' como se llaman actualmente), tal y como su nombre indica, es un juego diseñado con propósitos educacionales o que, de forma incidental o secundaria, tiene valor educativo. Cualquier juego ya sea en forma de juego de mesa, de cartas, o ser un videojuego puede ser usado, con el enfoque adecuado, en un ambiente educacional. Un juego educativo es un juego diseñado para enseñar a los humanos sobre una materia específica o destreza. De hecho, cuando los educadores, gobiernos y padres se dieron cuenta de la necesidad psicológica y los beneficios de aprender jugando, esta herramienta educacional se extendió masivamente. Los juegos son una herramienta interactiva que, al utilizarlos, nos enseñan objetivos, reglas, resolución de problemas, interacción, todo representado como una historia.

El juego siempre ha sido una herramienta de aprendizaje para enseñar conceptos o habilidades nuevas. Antiguamente se usaban las parábolas y las fábulas para promover el cambio social, incluso en la Edad Media se enseñaba a jugar al ajedrez con el fin de enseñar a usar tácticas en las guerras. Pero lamentablemente, no existe mucha información concerniente a ello. No fue hasta el siglo XIX que el hombre empezó a tomarse en serio el uso de los juegos como una manera de enseñar con la creación de los jardines de infancia por Friedrich Fröbel, que basaba el aprendizaje mediante el juego. Los niños jugaban encantados con sus simples juegos educativos: bloques, kits de coser y materiales para tejer.

\begin{figure}[H] 
	\begin{center}
		\includegraphics[scale=0.3]{ajedrez-edad-media.jpg}
	\end{center}
	\caption{Caballeros Templarios jugando al ajedrez, según el Libro de los Juegos (1283)}
	\label{fig:ajedrez-edad-media}
\end{figure}

Los juegos educativos fueron expandiéndose en diversas materias y creando juegos de mesas, cartas, etc. Pero no fue hasta la aparición de la tecnología en las casas, en la década de los 70, cuando Clark Abt propuso en su libro ``Serious Games''\cite{ccabt} una definición concreta de este tipo de juegos:

\emph{``Reducido a su esencia formal, un juego es una actividad entre dos o más personas con capacidad para tomar decisiones que buscan alcanzar unos objetivos dentro de un contexto limitado. Una definición más convencional es aquella en la que un juego es un contexto con reglas entre adversarios que intentan conseguir objetivos. Nos interesan los juegos serios porque tienen un propósito educativo explícito y cuidadosamente planeado, y porque no están pensados para ser jugados únicamente por diversión.''}

Aparte de esta definición, también se incluyó términos tales como ``juego educativo'', ``simuladores'', ``edutainment'' (entretenimiento educativo), etc. que años más tarde, se pondrían en práctica no solo en los juegos de mesa y cartas, sino también en la incipiente industria de los videojuegos. 

\begin{figure}[H] 
	\begin{center}
		\includegraphics[scale=0.7]{army-battlezone-atari.png}
	\end{center}
	\caption{Army Battlezone tenía unos gráficos superiores a su época al ser en un principio de uso militar}
	\label{fig:army-battlezone-atari}
\end{figure}

Army Battlezone \cita{army-battlezone} se considera el primer videojuego dentro de la categoría de juegos serios, un proyecto fallido liderado por Atari en 1980, el cual fue diseñado para usar el videojuego arcade Battlezone como entretenimiento militar. Algunos juegos triunfaron también en el sector de los ``edutainment'' como fue el caso de ¿Dónde está Carmen Sandiego en el mundo? \cita{carmen-sandiego} (Apple II, 1985) que acabó convirtiéndose en los 90 en una franquicia de juegos, series de televisión y libros; o más tarde, la saga EcoQuest \cita{eco-quest} de Sierra Online o el de la aventura gráfica de La Pantera Rosa en Misión Peligrosa \cita{pink-panther} (PC, 1996). Pero a pesar de los esfuerzos de muchas compañías como Disney o Nintendo, la mayoría de ellos fueron un fracaso tras otro. Los juegos de entretenimiento educativo, no eran rentables. 

\begin{figure}[H] 
	\begin{center}
		\includegraphics[scale=1]{carmen-sandiego-original.png}
	\end{center}
	\caption{¿Dónde está Carmen Sandiego en el mundo? fue uno de los pocos juegos educativos con éxito}
	\label{fig:carmen-sandiego}
\end{figure}

\begin{figure}[H] 
	\begin{center}
		\includegraphics[scale=0.7]{mario-is-missing.jpg}
	\end{center}
	\caption{La saga Mario is Missing (MS-DOS, 1992) fue uno de los intentos fallidos por parte de Nintendo de realizar juegos educativos}
	\label{fig:mario-is-missing}
\end{figure}

Así que según fueron creciendo las capacidades técnicas de los juegos para proporcionar escenarios realistas, el concepto de juegos serios tuvo que ser reexaminado a finales de la década de los 90 con el fin de reorientar el camino de los juegos educativos. Durante este tiempo, algunos estudiosos comenzaron a examinar la utilidad de los juegos para otros propósitos, contribuyendo al creciente interés por emplearlos con nuevos fines. Además, la capacidad de los juegos para contribuir a la formación se vio ampliada con el desarrollo de los juegos multijugador. 

En 2002, el Centro Internacional para Académicos Woodrow Wilson creó la Serious Games Initiative \cita{serious-game-initiative} con el fin de fomentar el desarrollo de juegos sobre temas políticos y de gestión. Otros grupos más especializados aparecieron después en 2004, como por ejemplo Games for Change \cita{games-for-change}, centrado en temas sociales y en cambio social, y Games for Health, sobre aplicaciones relacionados con la asistencia sanitaria. Pero no se llegó a actualizar el término de juego serio.

Hasta 2005, no se abordó este término de una forma actualizada y lógica. Mike Zyda escribió artículo publicado en la revista ``Computer'' de la IEEE Computer Society que llevaba por título ``From Visual Simulation to Virtual Reality to Games''\cite{mzynda}. Zyda define primero el término de qué es un juego y luego continúa a partir de aquí:

\begin{itemize}
	\item \negrita{Juego}: una prueba física o mental, llevada a cabo de acuerdo con unas reglas específicas, cuyo objetivo es divertir o recompensar al participante.
	\item \negrita{Videojuego}: una prueba mental, llevada a cabo frente a una computadora de acuerdo con ciertas reglas, cuyo fin es la diversión o esparcimiento, o ganar una apuesta.
	\item \negrita{Juego serio}: una prueba mental, de acuerdo con unas reglas específicas, que usa la diversión como modo de formación gubernamental o corporativo, con objetivos en el ámbito de la educación, sanidad, política pública y comunicación estratégica.
\end{itemize}

Fue ampliamente aceptada por el público dicha definición, aunque no es la única para el término de "juego serio", pero se entiende que hace referencia a juegos usados en ámbitos como la formación, la publicidad, la simulación o la educación. Definiciones alternativas incluyen conceptos propios de los juegos y las tecnologías, así como nociones provenientes de aplicaciones no relacionadas con el entretenimiento. Los juegos serios empiezan a incluir también hardware específico para videojuegos, como por ejemplo de los videojuegos para mejorar la salud y la forma física (véase Wii Fit \cita{wii-fit}).

Los videojuegos son una herramienta a tener en cuenta en la estimulación cognitivo afectiva, que favorecen el aprendizaje, la autoestima, potencian la creatividad y las habilidades digitales, al mismo tiempo que generan motivación y entretenimiento. Los videojuegos suponen una modalidad de enseñanza que debe ser aprovecha por la comunidad educativa, por la cantidad de elementos emocionales que integran, su estimulación sensorial y la posibilidad de inmersión a través de los ambientes virtuales en los que se desenvuelven.

\begin{figure}[H] 
	\begin{center}
		\includegraphics[scale=0.7]{brain-training.jpg}
	\end{center}
	\caption{Brain Training del Dr. Kawashima (Nintendo DS, 2005), un juego que estimulaba la mente para que pensáramos más rápido }
	\label{fig:brain-training}
\end{figure}

Los juegos serios están dirigidos a una gran variedad de público, desde estudiantes de educación primaria y secundaria a profesionales y consumidores. Los juegos serios pueden ser de cualquier género, usar cualquier tecnología de juegos y estar desarrollados para cualquier plataforma. Algunos lo consideran un tipo de entretenimiento educativo, aunque el grueso de la comunidad se resiste a este término.

Un juego serio puede ser una simulación con la apariencia de un juego, pero está relacionado con acontecimientos o procesos que nada tienen que ver con los juegos, como pueden ser las operaciones militares o empresariales. Los juegos están hechos para proporcionar un contexto de entretenimiento y autofortalecimiento con el que motivar, educar y entrenar a los jugadores. Otros objetivos de estos juegos son el marketing y la publicidad. Los grandes usuarios de los juegos serios parecen ser el gobierno de los Estados Unidos y los médicos. Otros sectores comerciales están también persiguiendo activamente el desarrollo de este tipo de herramientas.

\begin{figure}[H] 
	\begin{center}
		\includegraphics[scale=0.2]{simulador-vuelo-profesional.jpg}
	\end{center}
	\caption{Simulador de vuelo profesional para enseñar a los nuevos pilotos}
	\label{fig:simulador-vuelo}
\end{figure}

Actualmente existe una clasificación, más o menos aceptada pues no existe consenso oficial, de los juegos serios de acuerdo a su propósito:

\begin{itemize}
	\item \negrita{Advergames}: del inglés \emph{advertising} y \emph{game}, es decir, publicidad y juego, es la práctica de usar videojuegos para publicitar una marca, producto, organización o idea.
	
	\item \negrita{Edutainment}: este es un término que resulta de la unión de \emph{education} y \emph{entertainment}, es decir, educación y entretenimiento o diversión. Se aplica a los programas que enseñan mediante el uso de recursos lúdicos.
	
	\item \negrita{Aprendizaje basado en juegos}: del inglés \emph{educational game}, estos juegos tienen como objetivo mejorar el aprendizaje. Están diseñados en general manteniendo un equilibrio entre, por un lado, la materia y, por otro, la jugabilidad y la capacidad del jugador para retener y aplicar dicha materia en el mundo real. Este último tipo de juegos se utilizan en el mundo empresarial para mejorar las capacidades de los empleados en temas, atención al público y negociaciones.
	
	\item \negrita{Edumarket Games}: cuando un juego serio combina varios aspectos (por ejemplo, los propios del advergaming y del edutainment u otros relacionados con la prensa y la persuasión), se dice que la aplicación es un juego de tipo edumarket, término que resulta de la unión de \emph{education} y \emph{marketing}.
	
	\item \negrita{News Games}: son juegos periodísticos (del inglés \emph{news}, es decir, noticias) que informan sobre eventos recientes o expresan un comentario editorial.
	
	\item \negrita{Simuladores}: son juegos que se emplean para adquirir o ejercitar distintas habilidades o para enseñar comportamientos eficaces en el contexto de situaciones o condiciones simuladas. En la práctica, son muy usados los simuladores de conducción de vehículos (coches, trenes, aviones, etc.), los simuladores de gestión de compañías y los simuladores sobre negocios en general, que ayudan a desarrollar el pensamiento estratégico y enseñan a los usuarios los principios de la micro y macroeconomía y de la administración de empresas.
	\item \negrita{Juegos persuasivos}: del inglés \emph{persuasive games}, son juegos que se usan como tecnología de la persuasión para convencer a sus jugadores de que un concepto o idea está bien o mal.
	\item \negrita{Juegos organizativos dinámicos}: del inglés \emph{organizational-dynamic games}, son juegos que enseñan y reflejan la dinámica de las organizaciones a tres niveles: individual, de grupo y cultural.
	\item \negrita{Juegos para la salud}: del inglés \emph{games for health}, son juegos diseñados como terapia psicológica, o juegos para el entrenamiento cognitivo o la rehabilitación física.
	\item \negrita{Juegos artísticos}: del inglés \emph{art games}, son juegos usados para expresar ideas artísticas, o arte creado, utilizando como medio los videojuegos.
	\item \negrita{Militainment}: es un término de la unión de \emph{military} y \emph{entertainment}, es decir, militar y entretenimiento o diversión. Son juegos financiados por el ejército o que, de lo contrario, reproducen operaciones militares con un alto grado de exactitud. Lamentablemente, en su mayoría son privados para el público general dado el secretismo militar sobre para qué lo usan y cómo lo usan.
\end{itemize}

Cómo conclusión, podemos decir que si actualmente los juegos serios se han separado bastante de los videojuegos como medio de entretenimiento, aún siguen perdurando juegos de la gama de edutainment que intentan aunar estos dos campos. Pero lo que si podemos decir con certeza es que los primeros videojuegos educativos eran en su mayoría aventuras gráficas tal y como se ha visto en los ejemplos dados. No es difícil de saber el por qué, las aventuras gráficas, tal y cómo se describen en la sección anterior, son juegos dedicados a resolver problemas de lógica. Al recurrir más al esfuerzo mental que otros juegos que buscan los reflejos motores en los jugadores, es fácil realizar conexiones lógicas con las que enseñar datos concretos de forma que acabemos asociándolos a la realidad y aprendamos.

%\newpage
\section{Motivaciones}
\label{sec:motivaciones}
Comenzar a usar una tecnología desconocida como \programa{Unity3D}, aprender un nuevo lenguaje como C\# junto con la API de \programa{Unity3D} para este lenguaje, o cualquier otra, siempre entraña dificultad y, si le añadimos la documentación histórica para contrastar que toda la información que se presente tanto en esta memoria como en el videojuego final será verídica, aún más. La principal motivación para embarcarse en la confección de este proyecto, es la de la capacidad que existe en un videojuego tanto para entretener, transmitir sentimientos, ideas y emociones, como para ser una herramienta de apoyo a la enseñanza en prácticamente cualquier ámbito que se proponga la industria.

Diseño de Videojuegos era una asignatura optativa de tercer curso de la extinta titulación Ingeniería Técnica de Informática de Sistemas dentro de la Universidad de Cádiz. En dicha asignatura los alumnos se organizaban en grupos de tres alumnos con el objetivo de desarrollar un juego sencillo durante el cuatrimestre. Las únicas restricciones eran que el videojuego resultante debe ser completamente libre y que se había de utilizar la forja de código de RedIRIS \cita{rediris} junto a Subversion \cita{subversion} como sistema de control de versiones. Los alumnos escogían bibliotecas, usualmente libres, para el desarrollo de aplicaciones multimedia en dos dimensiones como libSDL \cita{sdl} o Gosu \cita{gosu}. Toda la documentación estaba en inglés, pero por suerte ya se publicó una excelente documentación en castellano en formato wiki llamada Wikijuegos \cita{wikijuegos} para libSDL.

Dicha asignatura era casi totalmente enfocada a aprender a programar y el diseño era algo más secundario, limitándose a unas pocas páginas que definieron a grandes rasgos los elementos que iban a desarrollarse en el juego.

Lamentablemente mucha, prácticamente toda la documentación sobre el diseño de juegos con una complejidad alta o mediana está en inglés. Poniendo por ejemplo el caso de documentos de diseño de aventuras gráficas, las cuales son orientadas más al contenido narrativo y varían el diseño general por uno más específico para puzzles, solo se pueden encontrar liberados documentos de puzzles y/o diseño en este idioma, siendo las más destacadas la de Grim Fandango \cita{grimfandango_doc}, las aventuras clásicas de AI Lowe \cita{ailowe_doc} y alguna que otra reciente como la de Resonance en foros de AdventureGameStudio \cita{resonance_doc}. Con todo esto, una de mis principales motivaciones es la de crear un documento de diseño extrapolable a todos los géneros, y aparte un documento de diseño de puzzles, todo esto en español y lo más completa posible, para que sirva de ejemplo para desarrolladores de videojuegos de habla hispana.

Por otra parte, dejando de lado el diseño, lo mismo se puede decir de ejemplos de código en español de aventuras gráficas. \programa{Unity3D} es una herramienta ideal para ello, pues es la más usada actualmente para iniciarse en el mundo del desarrollo de videojuegos, además de ser ampliamente usada por estudios por su versión gratuita y la variedad de plataformas en las que se puede compilar con un único código. Así pues, una motivación es la de generar un juego completo con el código documentado dividido en bloques reutilizables.

La ubicación de la Universidad de Cádiz y el pasado histórico su ciudad fue otra motivación, pues cuando empezaba con las dudas de sobre qué proyecto iba a realizar, se celebró el Bicentenario de la Constitución de 1812. Proporcionándome la idea de dar a conocer cómo es Cádiz y su historia del 1812, tanto para residentes de Cádiz, como españoles y extranjeros que nunca hayan visitado Cádiz.

Es posible resumir las motivaciones que me han llevado a desarrollar este proyecto en cubrir el vacío de documentación y la falta de proyectos completos en castellano, y el deseo personal de aprender a diseñar videojuegos.

%\newpage
\section{Objetivos}
\label{sec:objetivos}
El objetivo del proyecto viene extraído de forma lógica de las motivaciones (sección~\ref{sec:motivaciones}). Con \nombrejuego se pretende crear un juego ampliamente documentado y accesible que sirva de referencia para futuros desarrolladores de videojuegos españoles. Deberá debe cumplir con los siguientes requisitos:

\begin{itemize}
\item Servir de ejemplo lo más cercano a la realidad posible de lo que es un diseño medianamente complejo de un videojuego. Es muy habitual terminar de leer documentación, realizar diseños sencillos para pequeños juegos pero que a la hora de embarcarse a un proyecto con más complejidad es necesario organizarlo y esquematizarlo bien o acabaremos con un videojuego que no tiene ni pies ni cabeza. Para eso es imprescindible que se documente cada fase del desarrollo.
\item Crear un sistema intermediario entre el videojuego y el texto de los diálogos, de forma que, al ser independiente el texto, se pueda modificar sin interferir con el código, e incluso poder crear traducciones sin afectar en nada al mismo. Este sistema debe ser reutilizable y estar bien documentado.
\item Crear una Base de Datos con información histórica y un buscador dentro del juego para acceder a ella. El motivo de esto es el poder acceder a los datos de manera rápida y sencilla para poder seguir avanzando en el juego sin problemas.
\item Convertirse en un ejemplo de trabajo con un equipo multidisciplinar. En un videojuego profesional deben trabajar juntos desde unas pocas personas hasta varios centenares por lo que la coordinación entre profesionales de distintas áreas es muy importante. Será necesario encontrar artistas relacionados con la animación e ilustración, música y efectos de sonido.
\item No sólo debe ser útil a los interesados en el desarrollo sino también de cara al usuario final. \nombrejuego debe de ser intuitivo, divertido y útil para ayudar a asimilar datos, en este caso históricos.
\item Crear subsistemas independientes del videojuego reutilizables. Por supuesto, deben venir acompañados de la documentación pertinente: dependencias, instalación, uso y licencia.
\end{itemize}

\subsection{Sobre este documento}
La presente memoria de \negrita{Proyecto fin de Carrera} posee la siguiente estructura en capítulos y apéndices:

\begin{itemize}

\item En el capítulo~\ref{chap:introduccion} que actualmente nos ocupa, se hace una breve introducción sobre el contexto en el que nos situamos. Se ofrece una visión del mundo de la industria de videojuegos, principalmente independiente, se habla sobre las aventuras gráficas, que es el género en el que se va a basar \nombrejuego y la importancia de dicho género en el ámbito educativo. Finalmente se evidencia la necesidad de documentación en cuanto a diseño de videojuegos medianamente complejos en castellano y se adjuntan el resto de motivaciones y objetivos relacionados con el proyecto.

\item En el capítulo~\ref{chap:planificacion} se expone la organización temporal de todo el desarrollo a través de un diagrama de Gantt. Más adelante se comenta brevemente cada una de las tareas que conforman la planificación.

\item En el capítulo 3 se refleja el proceso de desarrollo para el videojuego \nombrejuego. Comenzando con el documento de diseño, el documento de puzzles, se sigue con la fase de análisis, diseño e implementación. Se hace especial hincapié en el sistema intermediario del texto con el juego, y la base de datos e introducción en el juego. Se finaliza con las pruebas realizadas.

\item En el capítulo 4 se hace una breve reflexión personal seguida de otra técnica tratando de aglutinar los conceptos aprendidos, la riqueza que ha proporcionado la experiencia y lo que podrá aportar tanto a usuarios como desarrolladores. Finalmente termina con un pequeño listado sobre las posibilidades de ampliación del proyecto.

\item En el apéndice Software utilizado se hace un repaso por todas las herramientas empleadas a lo largo del desarrollo del proyecto. Asimismo, se adjuntan comentarios y razones de uso.

\item En el apéndice Manual de usuario se incluye un completo manual de usuario para \nombrejuego. El documento contiene la introducción a la historia, personajes y lugares así como una completa guía de instalación en sistemas GNU/Linux y Windows. Posteriormente se detallan las mecánicas de juego y los controles en detalle. Finalmente se añade una completa guía para añadir traducciones adicionales a \nombrejuego usando el lenguaje XML.

\end{itemize}


%\newpage        
\chapter{Jugabilidad y mecánicas}
\label{chap:jugabilidad}
En esta sección entraremos más en detalle en lo que a las mecánicas de \nombrejuego se refiere. Se comentarán todas las características que forman parte de la jugabilidad, y se detallarán las acciones que podrá llevar a cabo el jugador dentro de una partida típica. Además de explicar de forma concisa la organización de los menús y su utilidad.

    \section{Jugabilidad}
        \subsection{Progresión del juego}
        La progresión del juego va definida por el número de localizaciones (nuevos escenarios) desbloqueados. Al principio solo contaremos con una única localización e iremos desbloqueando las demás según resolvamos puzzles.
        
        \subsection{Progresión de la dificultad}%Challenge Structure
        %Si bien los puzzles seguirán distintos métodos para su resolución, todos los puzzles serán presentados previamente por un personaje (sea el protagonista o un PNJ).
        Cada puzzle del juego tendrán distintos métodos de resolución, ya sea uno de colocar figuras en el mapa, otro de identificación de cuadros, uno de obtener objetos, etc. Aparte de eso, la dificultad irá en aumento por la cantidad de acciones, objetos necesarios, o áreas para visitar para la resolución de dicho puzzle.
        
        \subsection{Objetivos}
        El objetivo del juego es resolver el misterio del paradero del profesor desaparecido, y por consiguiente, que el protagonista apruebe el examen suspendido. Para ello, tendrá que resolver los puzzles, uno a uno, y finalmente resolver un pequeño cuestionario con las anécdotas históricas de cada puzzle.
        
        \subsection{Flujo de juego}%Play flow
        A lo largo de esta sección, se detallará el transcurso de una partida típica a \nombrejuego. Se comentarán los pasos que ha de seguir el \emph{Jugador} desde el arranque del juego hasta completar un puzzle y pasar al siguiente nivel. De esta forma, desentrañamos el funcionamiento exacto del juego. Más adelante se definen las mecánicas y el contenido de cada pantalla.
        
        El \emph{Jugador} inicia \nombrejuego y se le presenta el \emph{Menú Principal}. Si desea iniciar una partida el \emph{Jugador} seleccionará la opción \emph{Partida nueva}, o en el caso de tener una partida ya guardada, a la opción \emph{Cargar partida} (para saber más de los menús, véase la sección ~\ref{sec:diagrama-flujo-menus}).
        
        Una vez dentro del juego, veremos al personaje principal en un escenario. Dentro del cual, podremos interactuar con el escenario haciendo \emph{clicks} para ver o interactuar con sus elementos: ver lo que son, hablar con otros personajes, coger objetos, etc. Si estamos al inicio de un escenario nuevo, se sucederá una escena cinemática (de forma en la que el protagonista se mueva y hable sin requerir ninguna acción por parte del \emph{Jugador}), y un personaje de dicho escenario nos planteará un puzzle a resolver si queremos recibir una pista sobre el paradero del profesor. Según cómo hagamos dichas interacciones resolveremos el puzzle a resolver en dicho escenario y obtendremos un nuevo escenario a donde ir.
    
    %\newpage    
    \section{Controles}
        
        \begin{figure}[H] 
			\begin{center}
				\includegraphics[scale=0.5]{controles.png}
			\end{center}
			\caption{Controles del juego}
			\label{fig:controles}
		\end{figure}
    
    %\newpage    
    \section{Mecánicas}
        
        \subsection{Movimiento}
        En esta sección definiremos cómo se moverá el personaje principal que maneja el \emph{Jugador}, además de otros movimientos tales como trasladar un objeto de sitio.
            
            \subsubsection{Movimiento general}
            Para el movimiento general del personaje principal, el jugador solo se limitará en hacer click en alguna zona donde no haya obstáculos o encima de un objeto que se pueda examinar, de forma que el personaje principal se acerque para examinarlo más detenidamente.
            
            \subsubsection{Otros movimientos}
            Los otros personajes, no se moverán o simplemente se moverán unos pasos, siento su ruta únicamente una línea recta de escasa distancia.
            
        \subsection{Objetos}
        En los escenarios en los que transcurre el juego, habrá varios objetos con los que el \emph{Jugador} podrá interactuar con ellos.
            
            \subsubsection{Describiendo objetos}
            Al mover el ratón por la pantalla, el icono del ratón cambiará a otro si está encima de un objeto. Si el \emph{Jugador} pulsa \emph{click} izquierdo, el protagonista se moverá hasta la posición más cercana a dicho objeto y le explicará al \emph{Jugador} qué es y sus impresiones propias sobre él.
            
            \subsubsection{Cogiendo objetos}
            Si el objeto es adquirible por nuestro protagonista, de la misma manera que pasamos el ratón sobre el objeto y cambia el icono del ratón al estar sobre un  objeto, en este caso hacemos \emph{click} derecho. De esta manera el protagonista se moverá hasta la posición más cercana a dicho objeto y lo cogerá, seguido de un sonido que indique el haber metido el objeto en el inventario.
            
            \subsubsection{Moviendo objetos}
            En \nombrejuego realmente no podremos mover \emph{in situ} los objetos dentro del escenario. Para poder moverlos, primero tendremos que introducir el objeto en el inventario, luego seleccionarlo dentro del inventario con \emph{click} izquierdo, y finalmente usar el objeto en el sitio concreto del escenario donde se pueda utilizar con \emph{click} derecho.
            
        \subsection{Acciones}
        Si bien casi toda las acciones se hacen de manera similar a lo descrito en la sección anterior, existen algunas diferencias que hay que puntualizarlas. Pues no todo es simplemente describir, coger y mover objetos en \nombrejuego.
        
            \subsubsection{Interruptores y botones}
            Para accionar un interruptor o botón, simplemente hay que posicionar el ratón sobre él y ver cómo cambiar el icono de este. Estando así, simplemente hay que hacer \emph{click} derecho con el ratón, el protagonista se moverá hasta donde está el interruptor o botón y lo accionará. De manera opcional, el protagonista puede decirnos si ha pasado algo en concreto al accionarlo.
            
            \subsubsection{Coger, llevar y soltar}
            En \nombrejuego solo podremos coger, llevar objetos de la manera especificada en la sección de Objetos. Soltar los objetos no será posible hasta que se hayan usado en el puzzle que los requiera.
            
            \subsubsection{Hablar}
            En el juego, podremos hablar con los otros personajes que se encuentren en el escenario. Para ello, tendremos que posicionar el ratón encima de un personaje. El icono cambiará como en las veces citadas anteriormente, y tendremos que hacer \emph{click} izquierdo. Así el protagonista se moverá a la posición más cercana que se sitúe al frente del personaje y aparecerá un diálogo entre ellos.
            
            \subsubsection{Leer}
            En \nombrejuego habrá libros y carteles para leer. La mecánica es la misma que se usa para describir un objeto, dado que los libros y carteles son objetos. 
         
	%\newpage
	\section{Inteligencia Artificial}
	En la mayoría de los videojuegos es uno de los puntos más importante, pues ella rige el comportamiento de todos los personajes, e incluso de objetos. Un enemigo que sigue una ruta y te persiga si te ve, un semáforo que cambia de luces y los coches se mueven con ello, todo esto y mucho más es gracias a las inteligencia artificial. Existen muchos y variados algoritmos para realizar distintos comportamientos dependiendo de si buscas que un personaje haga una ruta, una estrategia, o que actúen a modo de colmena los personajes que aparecen el juego, y más aún que desconozco. Sin embargo, en las aventuras gráficas no se suele hacer un gran uso de estas pues los personajes no suelen moverse de su sitio, y las conversaciones son lineales. Pero si hay un punto que a veces usa una inteligencia artificial, cuando queremos que el personaje que lleva el \emph{Jugador} se mueva en un escenario con obstáculos.
        
            \subsection{Cálculo de la ruta del movimiento del personaje principal}
             Cuando el \emph{Jugador} quiera mover al personaje principal y haga \emph{click} en el escenario para ello. Pueden pasar varias cosas según la distancia que hay entre el punto al que el \emph{Jugador} quiere ir, y si hay obstáculos en el camino para llegar a ese punto.
             
             Si la distancia fuera corta, en línea recta y sin obstáculos, el movimiento se calcularía como una línea recta directa.
             
             En el caso de que hubiera obstáculos pero fuera posible llegar desde una ruta que no sea en línea recta, el personaje principal seguirá una ruta realizada con un algoritmo A\* o similar. De esta manera calcula la ruta más corta desde donde esta nuestro personaje principal hasta el punto donde queremos llevarlo, y finalmente lo ejecuta para llevar a cabo el movimiento.
            
            \subsection{Personajes No Jugables (PNJs) o \emph{Non Player Characters} (NPCs)}
        	Si algunas veces en las aventuras gráficas aparecen PNJs enemigos que persiguen al Protagonista o amistosos que estén dando vueltas por una ruta prefijada, en el caso de \nombrejuego no va a ser así. Los PNJs se quedarán en un sitio prefijado sin moverse y solo interactuarán con el Protagonista para iniciar una conversación. 
            
	%\newpage
	    \section{Físicas}
	    El juego poseerá físicas en 2D, osease, colisiones entre planos. De esta manera, el jugador solo podrá mover su personaje sobre ciertas áreas, saber cuándo estamos pulsando encima de un objeto o personaje con el que podemos interactuar, poder acceder al menú gracias a un menú gráfico, etc. No poseerá elementos gravitacionales y toda colisión tendrá valores discretos (no analógicos) para indicar si se ha producido una colisión o no.
	    
	    	\subsection{Sistema de colisiones}
	    	El sistema de colisiones será por cajas poligonales, en las que al colisionar unas con otras no podrán atravesarse mutuamente. Los elementos del juego que poseerán una caja de colisiones son:
	    	
	    	\begin{itemize}
	    	\item \negrita{Personaje principal:} Tendrá una caja de colisión en la parte inferior del \emph{sprite}. Esto es debido a que la parte superior de este puede superponerse parcialmente a un objeto que posea una caja de colisión y esté situado en un plano de fondo. Un ejemplo para esclarecer esto: el personaje principal puede estar encima de una pared parcialmente, pero sus pies no para evitar que camine sobre ella. A su vez el protagonista no podrá atravesar elementos en primer plano bajo ningún concepto, pero si puede aparecer detrás de ellos.
	    	\item \negrita{PNJs:} Su caja de colisión será idéntica a la del personaje principal. Pero esta servirá para que cuando el ratón se ponga encima del PNJ, el sistema lo detectará y contará como que el personaje principal puede hablar con dicho PNJ. El icono del ratón cambiará para mostrar que se puede realizar dicha acción.
	    	\item \negrita{Objetos:} Los objetos tendrán otra caja de colisión que no podrá atravesarse, pero esta estará más enfocada a que colisione con el ratón. Pues cuando este se ponga encima del objeto, se detectará la colisión y contará como que se puede examinar o interactuar con dicho objeto. El icono del ratón cambiará para mostrar que se puede realizar dicha acción.
	    	\item \negrita{Obstáculos:} Poseerán cajas de colisiones en la que su única función será el de no poder ser atravesados por el personaje principal. 
	    	\end{itemize}

%\newpage            
\chapter{Interfaz}
\label{chap:interfaz}
En un videojuego, la interfaz es esencial para transmitir información al jugador tanto dentro del juego para poder interactuar con dicho mundo, como en los menús para modificar elementos externos del juego tales como guardar una partida, cambiar especificaciones de la visualización del juego, etc. Por ello, en esta sección se especificará con detalle cada una de las pantallas y los menú que componen \nombrejuego. Además, se indicarán las transiciones entre ellas así como la utilidad de cada elemento de la GUI (\emph{Graphical User Interface}) o interfaz en español. Las imágenes adjuntas son bocetos que ilustran los componentes que debe contener cada pantalla, no obstante, los artistas podrán hacer cambios en la apariencia y disposición de los elementos si así lo consideran oportuno.
    
    \newpage
    \section{HUD (\emph{Heads-Up Display})}
    El HUD es el elemento de la GUI o interfaz diseñado para mostrar información al \emph{Jugador} dentro del juego. Usualmente está para indicar la vida, munición, armas y acciones a poder realizar. En el caso de las aventuras gráficas, el HUD se usaba principalmente para mostrar el inventario de objetos del que dispone el jugador y de las acciones que este puede realizar, además de algún icono para poder acceder a otros menús.
    
    En \nombrejuego al ser una aventura gráfica, su premisa es similar a la comentada en el párrafo anterior:
    
    \begin{figure}[H] 
	     \begin{center}
	         \includegraphics[scale=0.7]{dentro_del_juego.png}
	     \end{center}
	     \caption{Boceto del HUD dentro del juego}
	     \label{fig:hud-dentro-juego}
	\end{figure}
	
	Lista y descripción de todos sus componentes:
	\begin{itemize}
	\item \negrita{Botón Opciones}: al pulsarlo nos lleva a la pantalla \emph{Menú Opciones}.
	\item \negrita{Ratón del juego}: el icono del ratón dentro del juego irá cambiando según se coloque encima de un elemento con el que se pueda interaccionar.
	\item \negrita{Lista de objetos}: lista con los objetos que tiene el \emph{Jugador} en posesión hasta el momento. Para cada objeto se muestra una imagen y el nombre.
	\item \negrita{Pantalla de Juego}: Escenario 2D del juego. 
	\end{itemize}
    
    \newpage
    \label{sec:diagrama-flujo-menus}
        \section{Diagrama de flujo}
        En todo juego, los menús necesitan de una jerarquía u orden a la hora de aparecer en pantalla. Para ello, normalmente se usa un diagrama que establece de manera visual el orden y acceso a los distintos menús durante el juego. Normalmente en un juego pequeño, dónde solo hay uno o dos menús, se puede hacer innecesario. Pero en cuanto intentemos realizar un juego que necesite más de cuatro menús diferentes, se hace inviable el no esquematizar el orden de estos. No solo para el programador, sino para el jugador, pues el tener que ir buscando el menú específico de forma complicada, hará que dicho jugador se canse rápidamente del juego y consecuentemente abandonarlo. El siguiente diagrama de estados muestra las pantallas presentes a lo largo de \nombrejuego y las transiciones entre ellas.
        
        \begin{figure}[H] 
                \begin{center}
                    \includegraphics[scale=0.45]{diagrama_flujo_menu.png}
                \end{center}
                \caption{Diagrama de flujo de pantallas en el juego}
                \label{fig:diagrama-flujo-menu}
            \end{figure}
        
        \newpage
        \section{Descripciones de las pantallas}
        En esta sección nos centraremos en describir de manera individual todas las pantallas que hacen aparición en \nombrejuego.
        
        %\newpage
            \subsection{Menú principal}
            A continuación el boceto de la pantalla de \emph{Menú Principal} :
            
            \begin{figure}[H] 
                \begin{center}
                    \includegraphics[scale=0.7]{men_principal.png}
                \end{center}
                \caption{Boceto del Menú Principal}
                \label{fig:menu-principal}
            \end{figure}
            
            Lista y descripción de todos sus componentes:
            \begin{itemize}
            \item \negrita{Botón Nueva Partida}: al pulsarlo lleva a la pantalla de \emph{Crear Partida}.
            \item \negrita{Botón Cargar Partida}: al pulsarlo lleva a la pantalla de \emph{Cargar Partida}.
            \item \negrita{Botón Créditos}: al pulsarlo nos lleva a la pantalla \emph{Créditos}.
            \item \negrita{Botón Salir}: al pulsarlo nos lleva de vuelta al Sistema Operativo.
            \item \negrita{Protagonista y monumento}: Una ilustración del protagonista de \nombrejuego situado en un monumento emblemático del 1812.
            \end{itemize}
            
            \newpage
            \subsection{Cargar Partida}
            A continuación el boceto de la pantalla de \emph{Cargar Partida} :
            
            \begin{figure}[H] 
                \begin{center}
                    \includegraphics[scale=0.7]{cargar_partida.png}
                \end{center}
                \caption{Boceto de la pantalla de Cargar Partida}
                \label{fig:cargar-partida}
            \end{figure}
            
            Lista y descripción de todos sus componentes:
            \begin{itemize}
            \item \negrita{Lista de partidas}: lista con las partidas guardadas hasta el momento. Se muestra el día y la hora de la última actualización de esa partida.
            \item \negrita{Botón Cargar}: al pulsarlo coge la partida seleccionada y carga la pantalla de juego tal y como la dejamos.
            \item \negrita{Botón Borrar}: al pulsarlo nos dará un mensaje de aviso de si verdaderamente queremos borrar la partida, si le damos a sí nos borrará la partida, si elegimos la otra opción desaparecerá el mensaje de aviso sin realizar ningún borrado.
            \item \negrita{Botón Volver al menú}: al pulsarlo volvemos al \emph{Menú Principal}.
            \end{itemize}
            
            \newpage
            \subsection{Créditos}
            A continuación el boceto de la pantalla de \emph{Créditos} :
            
            \begin{figure}[H] 
                \begin{center}
                    \includegraphics[scale=0.7]{crditos.png}
                \end{center}
                \caption{Boceto de la pantalla de Créditos}
                \label{fig:creditos}
            \end{figure}
            
            Lista y descripción de todos sus componentes:
            \begin{itemize}
            \item \negrita{Panel}: texto con los componentes del equipo de desarrollo.
            \item \negrita{Botón Volver al menú}: al pulsarlo volvemos al \emph{Menú Principal}.
            \end{itemize}
            
            \newpage
            \subsection{Menú Opciones}
        	A continuación el boceto de la pantalla del \emph{Menú de Opciones} :
	         
	         \begin{figure}[H] 
	             \begin{center}
	                 \includegraphics[scale=0.7]{men_opciones.png}
	             \end{center}
	             \caption{Boceto de la pantalla del Menú de Opciones}
	             \label{fig:menu-opciones}
	         \end{figure}
	         
	         Lista y descripción de todos sus componentes:
	         \begin{itemize}
	         \item \negrita{Icono Mapa}: al pulsarlo vamos a la pantalla \emph{Mapa}. 
	         \item \negrita{Icono Gadipedia}: al pulsarlo vamos a la pantalla \emph{Gadipedia}. 
	         \item \negrita{Icono Notas}: al pulsarlo vamos a la pantalla \emph{Notas}.
	         \item \negrita{Icono Partidas Guardadas}: al pulsarlo vamos a la pantalla \emph{Partidas Guardadas}.
	         \item \negrita{Icono Mapa}: al pulsarlo vamos a la pantalla \emph{Ajustes}.
	         \item \negrita{Icono Salir}: al pulsarlo nos saldrá un mensaje que nos preguntará si queremos salir del juego y que todo lo que no haya sido guardado se perderá, si le damos a sí volvemos al \emph{Menú Principal}, si elegimos la otra opción el mensaje desaparecerá sin realizar ninguna acción.    
	         \item \negrita{Botón Volver al menú}: al pulsarlo volvemos a la pantalla de juego.
	         \item \negrita{Imágenes \emph{smartphone} y mano}: son las imágenes del fondo de la pantalla, simulando que el personaje principal usa su \emph{smartphone} para interactuar con los diferentes menús dentro del juego.
	         \end{itemize}
        
            \newpage
            \subsection{Mapa}
            A continuación el boceto de la pantalla \emph{Mapa} :
            
            \begin{figure}[H]
            	\begin{center}
            		\includegraphics[scale=0.7]{mapa.png}
            	\end{center}
            	\caption{Boceto de la pantalla Mapa}
            	\label{fig:mapa}
            \end{figure}
            
            Lista y descripción de todos sus componentes:
            \begin{itemize}
            \item \negrita{Panel Mapa}: Panel con una imagen de las calles del centro de Cádiz y sus respectivos nombres.
            \item \negrita{Iconos de Sitio}: Muestran un lugar al que se puede visitar dentro del juego, irá con el nombre del lugar. Al pulsarlo saldrá un mensaje preguntándonos si queremos ir a esa localización, si le decimos que sí volveremos a la pantalla de juego pero en la ubicación especificada, si elegimos la otra opción el mensaje desaparecerá sin realizar ninguna acción.
            \item \negrita{Icono Volver al Menú}: al pulsarlo volvemos a la pantalla \emph{Menú Opciones}.
            \item \negrita{Botón Volver al juego}: al pulsarlo volvemos a la pantalla de juego.
            
            \end{itemize}
            
            \newpage
            \subsection{Partidas guardadas dentro del juego}
            A continuación el boceto de la pantalla de \emph{Partidas Guardadas} dentro del juego :
            
            \begin{figure}[H] 
                \begin{center}
                    \includegraphics[scale=0.7]{cargar_partida_dentro_del_juego.png}
                \end{center}
                \caption{Boceto de la pantalla de Cargar Partida}
                \label{fig:cargar-partida-dentro-juego}
            \end{figure}
            
            Lista y descripción de todos sus componentes:
            \begin{itemize}
            \item \negrita{Crear partida nueva}: etiqueta y caja de texto para introducir el nombre de la nueva partida.
            \item \negrita{Botón Crear}: crea una nueva partida con el nombre que contiene la caja de texto. Si el perfil existe o no se ha introducido un nombre, una ventana con un mensaje aparecerá indicando el error.
            \item \negrita{Lista de partidas}: lista con las partidas guardadas hasta el momento. Se muestra el día y la hora de la última actualización de esa partida.
            \item \negrita{Icono Volver al Menú}: al pulsarlo volvemos a la pantalla \emph{Menú Opciones}.
            \item \negrita{Botón Volver al juego}: al pulsarlo volvemos a la pantalla de juego.
            \end{itemize}
            
            \newpage
            \subsection{Gadipedia}
            A continuación el boceto de la pantalla \emph{Gadipedia} :
            
            \begin{figure}[H] 
                \begin{center}
                    \includegraphics[scale=0.7]{gadipedia.png}
                \end{center}
                \caption{Boceto de la pantalla de Gadipedia}
                \label{fig:gadipedia}
            \end{figure}
            
            Lista y descripción de todos sus componentes:
            \begin{itemize}
            \item \negrita{Caja de texto}: caja de texto donde introducimos las palabras clave de la información que queramos saber de la historia de Cádiz de 1812.
            \item \negrita{Botón Buscar}: al pulsarlo buscará la información correspondiente a las palabras clave. Dependiendo de las palabras clave podrá pasar distintas cosas: que el buscador no encuentre ninguna información y muestre en una ventana emergente que no ha habido resultados, que las palabras clave sean ambigüas y muestre en una ventana emergente que pruebes a realizar la búsqueda con otros términos menos ambigüos, o que la búsqueda haya sido con éxito y las mostrará en el bloque de texto debajo del cuadro de búsqueda.
            \item \negrita{Bloque de texto e información}: bloque de texto informativo sobre un aspecto específico del Cádiz de 1812, podrá incluir fotos en un lado.
            \item \negrita{Icono Volver al Menú}: al pulsarlo volvemos a la pantalla \emph{Menú Opciones}.
            \item \negrita{Botón Volver al juego}: al pulsarlo volvemos a la pantalla de juego.
            \end{itemize}
            
            \newpage
            \subsection{Notas}
            A continuación el boceto de la pantalla \emph{Notas} :
            
            \begin{figure}[H] 
            	\begin{center}
            		\includegraphics[scale=0.7]{notas.png}
            	\end{center}
            	\caption{Boceto de la pantalla de Notas}
            	\label{fig:notas}
            \end{figure}
            
            Lista y descripción de todos sus componentes:
            \begin{itemize}
            	\item \negrita{Lista de notas}: lista con las notas que relatan los sucesos importantes que han ocurrido en la partida hasta el momento. Se muestra el nombre de la nota, que suele es una frase corta resumiendo el suceso.
            	\item \negrita{Bloque de texto}: bloque de texto informativo más descriptivo sobre el suceso importante.
            	\item \negrita{Icono Volver al Menú}: al pulsarlo volvemos a la pantalla \emph{Menú Opciones}.
            	\item \negrita{Botón Volver al juego}: al pulsarlo volvemos a la pantalla de juego.
            \end{itemize}
            
            \newpage
            \subsection{Ajustes}
            A continuación el boceto de la pantalla \emph{Ajustes} :
            
            \begin{figure}[H] 
            	\begin{center}
            		\includegraphics[scale=0.7]{ajustes.png}
            	\end{center}
            	\caption{Boceto de la pantalla de Ajustes}
            	\label{fig:ajustes}
            \end{figure}
            
            Lista y descripción de todos sus componentes:
            \begin{itemize}
            	\item \negrita{Lista despegable de idiomas}: lista despegable con los idiomas disponibles para el juego.
            	\item \negrita{Lista despegable de resolución}: lista despegable con las resoluciones de pantalla disponibles para el juego.
            	\item \negrita{Otros}: bloque destinado a futuros ajustes que puedan meterse en el videojuego.
            	\item \negrita{Icono Volver al Menú}: al pulsarlo volvemos a la pantalla \emph{Menú Opciones}.
            	\item \negrita{Botón Volver al juego}: al pulsarlo volvemos a la pantalla de juego.
            \end{itemize}
            
            %\newpage
            %\subsection{Salir}
        %\subsection{Trucos y \emph{Easter Eggs}}
        %Por ahora no hay intención de introducir trucos y \emph{Easter Eggs}.

%\newpage        
\chapter{Argumento y personajes}%Story Setting and Character
\label{chap:argumento}
En este apartado detallaremos el argumento y el guión del juego. Al ser \nombrejuego un videojuego del género de las aventuras gráficas, estos tienen más importancia, pues a través de la narración vamos conociendo a los personajes y a cómo resolver los puzzles que nos deparará en el juego. 
    \section{Argumento y narrativa}
        \subsection{Trasfondo argumental}
        Hace un par de días, el Protagonista ha suspendido un examen con la infame nota de un 4,9. Ante tal nota, el Protagonista se dispone ir a la revisión del examen al despacho del profesor. Lamentablemente el profesor no se encuentra allí, pero cómo nuestro Protagonista es un cabezota, hará todo lo posible para encontrar al profesor y convencerle de que merece el aprobado.
        
        \subsection{Elementos clave}
        \subsection{Secuencias cinemáticas}
        En esta sección detallaremos las secuencias cinemáticas que habrá en \nombrejuego. Estos son momentos del juego en el que el \emph{Jugador} no podrá controlar al Protagonista y el juego tomará control de la situación. En juegos grandes se suele invertir una cantidad enorme en hacer las secuencias cinemáticas lo más vistosas posibles empleando en su mayoría vídeos. Juegos más pequeños prefieren evitar estas escenas por el enorme gasto que suponen, y se las ingenian en hacerlos de otras maneras.
        
        En \nombrejuego al ser también un juego de pequeña envergadura, la cantidad de secuencias cinemáticas del juego es mínima y sólo habrá una al principio de entrar en un nuevo escenario, y otra después de resolver un puzzle. Por ahora solo hay 
            
    \section{Mundo del juego}
        \subsection{Estilo visual general}%General Look and Feel
        \subsection{Primer escenario}
        \subsection{Segundo escenario}
        %...
        
    \section{Personajes}
    En esta sección definiremos brevemente a los personajes que intervendrán en el juego. Servirá sobretodo a la hora de definir las personalidades y reflejarlas en la manera de hablar de estos en el guión.
    
        \subsection{Protagonista}
            \subsubsection{Trasfondo}
            Estudiante del Grado de Historia en la Universidad de Cádiz, ha suspendido un examen con un 4,9 y hará todo lo posible para convencer al profesor de que merece el aprobado.
            \subsubsection{Personalidad}
            De personalidad humorística y despreocupada, con cierta aversión al profesor que le ha suspendido.
            \subsubsection{Aspecto}
            Pelo negro y desaliñado, con barbita dejada tal y cómo se lleva ahora entre los universitarios. Con sudadera y vaqueros.
                
        \subsection{Profesor}
            \subsubsection{Trasfondo}
            Profesor de una asignatura en el Grado de Historia en la Universidad de Cádiz, no se sabe donde está.
            \subsubsection{Personalidad}
            Despistado y afable, pero es muy estricto con los alumnos.
            \subsubsection{Aspecto}
            Típico profesor mayor con barba y traje.
            
        \subsection{Bibliotecaria}
            \subsubsection{Trasfondo}
            Es la bibliotecaria de la Facultad de Filosofía y Letras desde hace muchos años, ya los años les pasa factura y se olvida de las cosas o no puede llevar los libros con tanta facilidad.
            \subsubsection{Personalidad}
            Olvidadiza y muy estricta con las normas de la biblioteca, sobretodo con la de mantener el silencio.
            \subsubsection{Aspecto}
            Mujer mayor con gafas de lectura antigua.
    
    Estos son los personajes definidos por ahora, más adelante se irán añadiendo más y así sucesivamente hasta completar con todo el elenco de personajes que harán su aparición en el juego.
        %...

%\newpage        
\chapter{Niveles}
\label{chap:niveles}
%Los niveles en este juego, serán los distintos escenarios donde el Jugador pueda moverse. \nombrejuego tiene pocos niveles por las limitaciones de tiempo y dinero que implica hacer un proyecto final para una carrera universitaria, pero todos estos están esquematizados para esclarecer todo lo que contiene y sucede dentro de dicho nivel.

Los niveles son los distintos escenarios por los que un \emph{Jugador} tendrá que pasar para poder avanzar en el videojuego, de forma que \negrita{siempre} hay que resolver el problema que nos supone el nivel actual para poder llegar al siguiente, y así sucesivamente hasta que acabemos el juego. Estos niveles o escenarios pueden ser pantallas estáticas tales como un nivel del Tetris, o un escenario donde el \emph{Jugador} pueda moverse de un lado para otro. En este segundo caso pueden existir dos variantes, que el \emph{Jugador} pueda recorrer el nivel con una vista en primera persona, o en tercera persona teniendo el \emph{Jugador} que mover un personaje para recorrerlo.

En cualquier caso, el diseño de los niveles es \negrita{vital} para que el jugador disfrute de la experiencia de superar el videojuego en concreto que vayamos a crear. Hay que lograr que nunca haya momentos aburridos y a su vez que no haya demasiados momentos de tensión que acaben en frustración para el \emph{Jugador}, y que acaben consiguiendo que este lo deje y nunca termine el juego. De hecho, aunque un juego pueda poseer buenas e interesantes mecánicas, los \emph{Jugadores} no las verán hasta que entren en juego en algún nivel e interactuen con este.

%En el caso de aventuras gráficas, concretamente \nombrejuego que es el juego a desarrollar aquí, sus niveles van divididos en distintos escenarios a recorrer. En estos hay que buscar objetos o sitios donde poder interactuar con el escenario u otros personajes, de forma que si hacemos la combinación correcta, logramos resolver el problema que nos han propuesto y así seguir avanzando hacia nuestro objetivo. En resumen: 

En el caso de aventuras gráficas, concretamente \nombrejuego que es el juego a desarrollar aquí, sus niveles van divididos en distintos escenarios a recorrer. De forma concisa, estos son los pasos que hay que seguir en el diseño de los niveles en este tipo de género:

\begin{enumerate}
	\item Al inicio del juego, nos plantean en el argumento un problema u objetivo a conseguir.
	\item Para poder conseguir el objetivo principal, hay que dividirlo en distintos sub-objetivos.
	\item Cada sub-objetivo se resuelven con la combinación de acciones y objetos adecuadas.
	\item Una vez resuelto dicho sub-objetivo, obtenemos nuevos niveles o escenarios a visitar.
	\item En los escenarios nuevos nos proporcionarán un nuevo sub-objetivo que nos acercará al objetivo principal del videojuego.
	\item Se repite el proceso de sub-objetivos hasta que finalmente recompensa a los \emph{Jugadores} con la obtención del objetivo principal del juego.
\end{enumerate}

Como véis, en las aventuras gráficas, muchas veces hablar de puzzles es sinónimo de hablar de niveles. En un principio, en \nombrejuego se tomó la decisión inicial de separar la parte de los puzzes (niveles) en otro documento mejor organizado. No se descarta en un futuro incluir dicha parte en este documento, pero para la versión del documento de diseño que nos ocupa, este capítulo sólo se quedará como orientativo. 

Las secciones siguientes se explicarán de forma breve cuál es su propósito en el juego, y de cómo podríamos escribirlas.


    \section{Nivel 1}
    Este es el nivel donde comenzamos el juego. Suele ser más simple y largo que otros, pues se usa principalmente para explicarle al \emph{Jugador} de qué va el juego y cuál es el objetivo principal de este, además de proporcional diversos elementos con los que el \emph{Jugador} empiece a familiarizarse con las mecánicas básicas del juego. 
        \subsection{Sinopsis}
        Aquí vendría un resumen de lo que va a pasar en este nivel, de sus características y elementos más destacados.
        \subsection{Introducción}
        Aquí deberíamos escribir qué es lo que ha sucedido antes de empezar el nivel y los hechos que pueden influir en el transcurso de este.  
        \subsection{Objetivos}
        Esto es lo antes comentado con los sub-objetivos del juego, aquí hay que decir qué es lo que el \emph{Jugador} tiene que lograr hacer en este nivel.
        \subsection{Descripción física}
        Tal y como indica el nombre, aquí incluiríamos una descripción detallada del nivel y de sus elementos.
        \subsection{Mapa}
        En el caso de que el nivel fuera de gran extensión, o que el nivel perteneciera a un lugar concreto dentro del mundo en el que se va a desarrollar el juego, se tendrían que incluir aquí imágenes diversas explicando los puntos de referencia o elementos clave del nivel. Así tanto para que el que se encarga de diseñar los niveles tenga claro donde colocar los elementos dentro del nivel, como para que el \emph{Jugador} tenga puntos de referencia para orientarse dentro del nivel.
        \subsection{Camino crítico}
        En los niveles, pueden haber distintas maneras de obtener los elementos clave para conseguir el objetivo del nivel. Aquí habría que detallar todas las rutas diferentes que un \emph{Jugador} puede seguir a través del nivel para conseguir el objetivo. Aunque hay que puntualizar, que a veces las rutas de resolución de un nivel pueden ser casi infinitas, en ese caso, solo tendríamos que escribir aquí las rutas principales que usará la mayoría de \emph{Jugadores}. 
        \subsection{Resolución del nivel}
        Aquí, a modo de guía de resolución de juego, tenemos que explicar cómo se resuelve el problema que nos presenta el nivel para lograr el objetivo.
        \subsection{Conclusión}
        Aquí comentaremos los sucesos que ocurran una vez finalizado el nivel y antes de que comience el siguiente.
        
    %\section{Nivel 2}
    %...

%%\newpage
%\chapter{Interfaz}
%
%    \section{Sistema visual}
%        \subsection{HUD}
%        \subsection{Menús}
%        
%    \section{Controles del sistema}
%        
%        \section{Audio}
%        \subsection{Música}
%        \subsection{Efectos de sonido}
%

%\newpage
%\chapter{Inteligencia Artificial}

%    \section{ \emph{Non Player Characters} (NPCs)}
        
%    \section{Jugador y sistema de colisiones}


        
%\newpage
\chapter{Apartado técnico}
\label{chap:tecnico}
En este apartado detallaremos las partes relacionadas con el desarrollo del videojuego como software, tales como para qué plataforma se va a desarrollar, y cómo y con qué se va a desarrollar \nombrejuego.

    \section{Hardware objetivo}
    \nombrejuego al ser una aventura gráfica, estará enfocado principalmente al PC. Tendrá que ser jugado con teclado y ratón. Sin embargo, si hubiera tiempo durante el desarrollo, no se descarta una conversión a Android.
    
    \section{Hardware y Software de desarrollo}
    Algunas veces, pueden ocurrir incidencias debido a las peculiaridades del hardware y software concretos con el que se desarrolla un juego. Por lo tanto, siempre es recomendable escribir las especificaciones tanto de hardware como de software en los que se desarrollan los videojuegos. En este caso, \nombrejuego se desarrollará en un PC con las siguientes especificaciones técnicas:
    
    \begin{itemize}
    \item Procesador: Intel(R) Core(TM) i7-2700K CPU @ 3.50GHz (8 CPUs), ~3.5GHz
    \item Memoria: 16384 MB RAM
    \item Tarjeta Gráfica: NVIDIA GeForce GTX 570 1GB
    \item Sistema Operativo: Windows 7 Enterprise 64 bis (6.1, compilación 7601)
    \end{itemize}
    
    \section{Procedimientos y estándares de desarrollo}
    Un proceso para el desarrollo de software, también denominado ciclo de vida del desarrollo de software es una estructura aplicada al desarrollo de un producto de software, en este caso un videojuego. Hay varios modelos a seguir para el establecimiento de un proceso para el desarrollo de software, cada uno de los cuales describe un enfoque diferente para diferentes actividades que tienen lugar durante el proceso. 
    
    En el caso de \nombrejuego, el proceso que se va a seguir es el de Modelo de desarrollo en Cascada. Este modelo se caracteriza en que se ordenan las etapas del proceso para el desarrollo de software, de tal forma que el inicio de cada etapa debe esperar a la finalización de la etapa anterior. Al final de cada etapa, el modelo está diseñado para llevar a cabo una revisión final, que se encarga de determinar si el proyecto está listo para avanzar a la siguiente fase.
    
    Las etapas son las siguientes:
    \begin{enumerate}
    \item \negrita{Análisis}: En esta fase se analizan las necesidades de los usuarios finales del software para determinar qué objetivos debe cubrir. De esta fase surge una memoria llamada documento de especificación de requisitos, que contiene la especificación completa de lo que debe hacer el sistema sin entrar en detalles internos.
Es importante señalar que en esta etapa se debe consensuar todo lo que se requiere del sistema y será aquello lo que seguirá en las siguientes etapas, no pudiéndose requerir nuevos resultados a mitad del proceso de elaboración del software de una manera.
    
    \item \negrita{Diseño}: Descompone y organiza el sistema en elementos que puedan elaborarse por separado, aprovechando las ventajas del desarrollo en equipo. Como resultado surge el SDD (Documento de Diseño del Software), que contiene la descripción de la estructura relacional global del sistema y la especificación de lo que debe hacer cada una de sus partes, así como la manera en que se combinan unas con otras.
Es conveniente distinguir entre diseño de alto nivel o arquitectónico y diseño detallado. El primero de ellos tiene como objetivo definir la estructura de la solución (una vez que la fase de análisis ha descrito el problema) identificando grandes módulos (conjuntos de funciones que van a estar asociadas) y sus relaciones. Con ello se define la arquitectura de la solución elegida. El segundo define los algoritmos empleados y la organización del código para comenzar la implementación.
    \item \negrita{Implementación}: Es la fase en donde se implementa el código fuente, haciendo uso de prototipos así como de pruebas y ensayos para corregir errores.
Dependiendo del lenguaje de programación y su versión se crean las bibliotecas y componentes reutilizables dentro del mismo proyecto para hacer que la programación sea un proceso mucho más rápido.

    \item \negrita{Pruebas}: Al final de la implementación, normalmente hay un proceso de verificación o de pruebas. De manera que los elementos, ya programados, se ensamblan para componer el sistema y se comprueba que funciona correctamente y que cumple con los requisitos.
    
    \item \negrita{Mantenimiento}: Una vez terminado el desarrollo del software, hay una última fase de duración indefinida que es la de mantenimiento. En esta simplemente es que durante el ciclo de vida útil del software, hay que reparar errores que se hayan pasado en la etapa de pruebas o irle añadiendo nuevas funcionalidades que el público objetivo pida y así alargar su vida útil.
    \end{enumerate}
    
    Si bien esta manera de desarrollar no puede ser la más óptima, pues en cuanto se descubra un fallo que pasó desapercibido en una etapa y haya que modificarlo, hay que invertir mucho esfuerzo y parar el desarrollo. Pero es la más sencilla para proyectos de poca envergadura y que tengan bajo coste. Además de fomentar las buenas prácticas al poner un orden a la hora del desarrollo, y hacer que pensemos cómo hacer las cosas antes de llevarlas a cabo. También es una buena opción para los proyectos que están orientados a documentos como el de \nombrejuego, por ello, este es el procedimiento de desarrollo elegido. 
    
    \section{\emph{Game Engine}}
    \programa{Unity3D} será el \emph{Game Engine} con el que se desarrollará \nombrejuego. Los motivos son porque es uno de los \emph{Game Engines} más usados en la industria del videojuego independiente por su capacidad de prototipado rápido y de bajo coste, además de ser uno de los que más gente usa para iniciarse en el mundo del desarrollo de los videojuegos. 
    
    \section{Lenguaje de programación}
    Unity3D acepta C\#, JavaScript y Boo como lenguajes de programación para sus scripts. \nombrejuego se desarrollará por C\# por sus similitudes con Java y C++, lenguajes enseñados en la carrera. Además de poseer una estructura clara para poder realizar Programación Orientada a Objetos (POO) y hacer uso de Estructuras de Datos. 


%\newpage    
\chapter{Apartado artístico}
\label{chap:artistico}
En este apartado detallaremos el estilo que deben seguir los gráficos en el juego a partir del arte conceptual, las guías de estilo, descripciones de cómo deben ser personajes y escenarios, escenas cinemáticas, etc.

    \section{Arte conceptual}
    El arte conceptual, en un videojuego, sirve para definir tanto personajes, escenarios como el estilo de los gráficos de los que va a tratar el juego. Normalmente en juegos que pretenden dar un toque distintivo mediante el apartado gráfico, ya sea por diferenciar su estilo propio o para causar determinadas emociones o ambientación, es muy importante. Pero en cambio, en juegos pequeños o que los gráficos sean más secundarios, no suelen tener tanta carga y se deja más al gusto personal del grafista. \nombrejuego un juego medianamente pequeño, su arte conceptual es escaso y se deja también en manos de la grafista. 
    %El único arte conceptual que incluiremos es el diseño inicial del Protagonista, por dar un ejemplo:
    
    \section{Guías de estilo}
    En los videojuegos siempre es necesario que haya una uniformidad en los gráficos, no sólo con el estilo visual, sino con el tamaño de los gráficos y de las pantallas. Asimismo, \nombrejuego propondrá las siguientes guías para los gráficos y que así la grafista pueda atenerse a unas normas a la hora de la realización de los gráficos. Dichas normas, son más de sugerencia para conseguir una uniformidad que una imposición hacia el grafista. Así que si la ocasión lo requiere, el grafista podría saltarse las guías siempre y cuando se consulte con los otros miembros del equipo relacionados con el arte del juego y se llegue a un consenso para ese cambio. Estas son las guías para \nombrejuego:
    
    \begin{itemize}
    \item Personajes: 64x40 píxeles de resolución original, al que luego se le aplicará un \emph{zoom} por dos veces.
    \item Escenarios: 320x240 píxeles de resolución original, al que luego se le aplicará un \emph{zoom} por dos veces.
    \item Pantalla: 1280x960 píxeles será la resolución original de la pantalla, a partir de ella se adaptará a las distintas resoluciones más comunes en los ordenadores.
    \item Objetos: 32x32 píxeles de resolución original, al que luego se le aplicará un \emph{zoom} por dos veces.
    \item Paleta de colores: preferiblemente 256 colores (8 bits) para simular mejor el ser un juego antiguo, pero si se diera el caso de necesitar más colores, no habría ninguna restricción en usarlos.
    \end{itemize}
    
    \section{Personajes}
    En esta sección haremos una pequeña descripción de los personajes que aparecen en el juego, de forma que esta sea útil para la grafista a la hora de saber cómo diseñar a estos. Los personajes por ahora son los siguientes:
    
    \begin{itemize}
    \item \negrita{Protagonista}: Un estudiante universitario con los pelos revueltos, con sudadera, vaqueros y zapatillas.
    \item \negrita{Profesor}: El típico profesor mayor de universidad, con barba y traje.
    \item \negrita{Bibliotecaria}: La típica bibliotecaria, una señora mayor con gafas de lectura colgadas en la pechera.
    \item \negrita{Escritor}: Un escritor afrancesado y borracho, sentado en la mesa de firma de sus libros.
    \item \negrita{Junta directiva}: Varios hombres sentados con traje de ejecutivo y mirada inquisitoria, sentados en una mesa larga como si fuera un juicio.
    \end{itemize}
    
    \section{Escenarios}
    En esta sección haremos una pequeña descripción de los escenarios que aparecen en el juego, de forma que esta sea útil para la grafista a la hora de saber cómo diseñar a estos. Los escenarios por ahora son los siguientes:
    
    \begin{itemize}
    \item \negrita{Pasillo de la Facultad}: Un pasillo de la Facultad con varias puertas que dan a los despachos de distintos profesores.
    \item \negrita{Despacho del profesor}: El despacho al que pertenece el profesor con el que quiere hablar el Protagonista, tiene dos mesas (son dos profesores por despacho), una estanteria con libros y otras cosas, un tablón de corcho con un mapa, carteles, etc.
    \end{itemize}
    
    Estos son los escenarios por ahora, se irán añadiendo más conforme se avance el desarrollo del juego.
    
    \section{Objetos}
    En esta sección detallaremos brevemente los objetos que el Protagonista puede llevar en el inventario. Los objetos por ahora son los siguientes:
    \begin{itemize}
    \item \negrita{Examen suspendido}: Una hoja escrita con un 4,9 en rojo.
    \item \negrita{Banderitas}: Una banderita española y varias francesas con una chincheta.
    \item \negrita{Nota de la biblioteca}: Una pequeña nota que indica que el profesor debe ir a la biblioteca.
    \end{itemize}
    
    \section{Escenas cinemáticas}
    En \nombrejuego prácticamente no hay escenas cinemáticas, pues se hará uso de las animaciones de los personajes para crear las escenas controladas por el juego. Solo habrá una escena cinemática como tal, que será el final del juego en el que se verán imágenes del Protagonista feliz por haber aprobado y el de otros personajes a modo de epílogo, mientras se muestras los créditos y los agradecimientos.
    
    \section{Miscelánea}
    En esta sección suelen entrar los gráficos que no tienen cabida en las otras secciones. Concretamente en \nombrejuego se necesitarán los siguientes gráficos extras:
    \begin{itemize}
    \item \emph{Smartphone} del Protagonista, negro y únicamente táctil.
    \item Botones de las aplicaciones del \emph{smartphone} del Protagonista.
    \item Mano del Protagonista que hace como que toca el \emph{smartphone}.
    \end{itemize}

    
%\newpage    
\chapter{Apéndice}
\label{chap:apendice}
 \section{Organización de los recursos dentro del proyecto}
    Que haya una buena organización de los recursos facilita mucho el trabajo a los desarrolladores a la hora de tener que implementarlo dentro del juego, y así no tener que buscarlo en un sinfín de carpetas desorganizadas. Por ello es importante establecer una jerarquía en el orden de los recursos y de las carpetas que lo contengan. La organización de los recursos dentro del repositorio de \nombrejuego es la siguiente:
    
    \begin{itemize}
    \item \negrita{Recursos}/
        \begin{itemize}
        \item Audio/
            \begin{itemize}
            \item Música/
            \item FX/
            \end{itemize}
        \item Gráficos/
            \begin{itemize}
            \item GUI/
                \begin{itemize}
                \item Pantalla 1/
                \item Pantalla 2/
                \item ...
                \end{itemize}
            \item HUD/
            \item Personajes/
                \begin{itemize}
                \item Personaje 1/
                \item Personaje 2/
                \item ...
                \end{itemize}
            \item Escenarios/
                \begin{itemize}
                \item Escenario 1/
                \item Escenario 2/
                \item ...
                \end{itemize}
            \item Objetos/
                \begin{itemize}
                \item Objeto 1/
                \item Objeto 2/
                \item ...
                \end{itemize}
            \end{itemize}
        \end{itemize}
    \end{itemize}
    
    \section{Lista de recursos}
    Aquí haremos una lista que se irá actualizando con los recursos audiovisuales necesarios para poder realizar \nombrejuego. Así no se hará ningún trabajo de más o de menos, y que en el caso de tener que realizar modificaciones, estas sean pocas.
    
        \subsection{Arte}
        Todos los recursos gráficos que necesitará el juego, por ahora son los comentados en las siguientes secciones y sub-secciones.
            \subsubsection{Lista de sprites y fondos}
            Los \emph{sprites} son el nombre con el que se llama a los dibujos pixelados con los que se hacen los gráficos 2D para los videojuegos, si bien los fondos son sprites es mejor llamarlos de manera diferente, el motivo principal es porque existe la diferencia de que un fondo y un sprite es su tamaño. Pues un fondo ocupa toda la pantalla, mientras que un sprite es mediano o pequeño, pero nunca ocupa la pantalla en su totalidad.
            Lista de sprites de personajes:
            \begin{itemize}
            \item Protagonista.
            \item Bibliotecaria.
            \item Profesor.
            \item Escritor fracasado en su mesa de firma de libros.
            \item Otro profesor.
            \item Vendedor de la tienda de recuerdos.
            \end{itemize}
            
            Lista de sprites de objetos:
            \begin{itemize}
            \item Examen con la nota de 4,9 puntos en rojo.
            \item Banderitas de España y Francia.
            \item Nota de biblioteca.
            \end{itemize}
            
            \subsubsection{Lista de animaciones}
            \begin{itemize}
            \item Protagonista caminando.
            \item Protagonista hablando.
            \item Protagonista cogiendo objeto del suelo y metiéndoselo en el bolsillo de la sudadera.
            \item Protagonista cogiendo un objeto a su nivel y metiéndoselo en el bolsillo de la sudadera.
            \item Protagonista mirando el móvil.
            \item Nota de papel cayendo en el suelo.
            \end{itemize}
            
            \subsubsection{Lista de efectos}
            En este juego no hay efectos.
            
            \subsubsection{Lista de arte de la interfaz}
            \begin{itemize}
            \item Icono de lupa para el ratón.
            \item Icono de lupa con borde blanco (u otro elemento que destaque) para indicar que hay un objeto con el que se puede interactuar.
            \item Icono de bocadillo de cómic para el ratón.
            \item Icono de flecha para cambiar o salir de escenario. 
            \item Fondo con un \emph{smartphone} negro y táctil, que se vea la mano del protagonista cómo si fuera a pulsar algo en el \emph{smartphone}.
            \item Icono de \emph{smartphone} pequeño para la pantalla de juego.
            \item Icono de aplicación de \emph{smartphone} de mapas como si fuera el Google Maps.
            \item Icono de aplicación de \emph{smartphone} de notas.
            \item Icono de aplicación de \emph{smartphone} para guardar partidas.
            \item Icono de aplicación de \emph{smartphone} de Gadipedia como si fuera la Wikipedia.
            \item Icono de aplicación de \emph{smartphone} de ajustes.
            \item Icono de aplicación de \emph{smartphone} para salir del juego y volver al menú principal.
            \item Fondo de \emph{smartphone} plano de varios colores.
            \item Botón de aplicación de \emph{smartphone} reutilizable (que se pueda alargar y ensanchar).
            \item Cuadro de texto de aplicación de \emph{smartphone} reutilizable (que se pueda alargar y ensanchar).
            \item Barra de scroll de aplicación de \emph{smartphone} reutilizable (que se pueda alargar).
            \item Mapa parcial y sencillo en dos colores planos de Cádiz para la aplicación de mapas (similar al Google Maps).
            \item Botón circular o banderita pequeña para indicar los lugares a los que podemos ir en el mapa.
            \end{itemize}
            
            \subsubsection{Lista de escenas cinemáticas}
            Solo habrá una escena cinemática, la de los créditos, que será realizada con la composición de otras imágenes estáticas y se irán cambiando. Dichas escenas cinemáticas, por ahora con posibilidad de modificación, son:
            \begin{itemize}
            \item Protagonista dándose la mano con el profesor.
            \item Protagonista estudiando en la biblioteca mientras pasa la bibliotecaria.
            \item ...
            \item Examen con un 4,9 tachado y un 5 en rojo, sobre una mesa.
            \end{itemize}
            
        \subsection{Sonido}
        Todos los recursos auditivos que necesitará el juego, por ahora son los comentados en las siguientes secciones y subsecciones.
            \subsubsection{Sonido ambiental}
            \begin{itemize}
            \item Racha de viento que sopla por la ventana del despacho.
            \item Murmullo de gente hablando y luego un carraspeo de garganta de la bibliotecaria (una señora mayor) para mandar a callar a la gente.
            \end{itemize}
            \subsubsection{Sonidos al realizar interacciones}
            \begin{itemize}
            \item Cerrar ventana.
            \item Coger papel del suelo.
            \item Coger chinchetas (u objetos semimetálico) del suelo.
            \end{itemize}
            \subsubsection{Sonidos de interfaz}
            \begin{itemize}
            \item Pulsación de móvil.
            \end{itemize}
            
        \subsection{Música}
        Todos los recursos musicales que necesitará el juego, por ahora son los comentados en las siguientes secciones y subsecciones.
            \subsubsection{Menú principal}
            Para esta pantalla una canción más elaborada vendría bien, no importa si es de tono desenfadado o más clásica.
            \subsubsection{Ambiente}
            Sería recomendable hacer una canción para cada escenario, todas ellas de tono desenfadado y con toque cómico. Que fueran cortas y se  puedan poner en bucle. 
            
            Aparte de esas canciones, vendría bien una que fuera más como un tono de victoria, que se pondría cada vez que se resuelva un puzzle y podemos ir al siguiente escenario.
            \subsubsection{Créditos}
            Lo mismo que en la de Menú principal, una canción más elaborada. La intención con esta canción es darle al \emph{Jugador} la sensación de satisfacción de haberse pasado el juego.
   


\end{document}
